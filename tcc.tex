\documentclass[oneside,a4paper,12pt]{normas-utf-tex}
\usepackage[alf,abnt-emphasize=bf,bibjustif,recuo=0cm,abnt-etal-cite=2]{abntcite}
\usepackage{graphicx,color}
\usepackage[brazil]{babel}
\usepackage[utf8]{inputenc}
\usepackage{amsthm,amsfonts}

\usepackage{amsmath}
\usepackage{makecell}

\graphicspath{{figures/}}


%=======================================================================
%Informações Gerais (a serem preenchidas pelo aluno)
%=======================================================================

%======Instituição e programa
\instituicao{Universidade Tecnológica Federal do Paraná}
\departamento{Departamento Acadêmico de Informática}
\programa{Bacharelado em Engenharia da Computação}

\orientador{Prof. Dr. Bogdan Tomoyuki Nassu}
\documento{Trabalho de Conclusão de Curso}

\cita{Kurpiel, Francisco Delmar}
\autor{Francisco Delmar Kurpiel}

\titulo{Localização de Placas Veiculares em Vídeo Usando Redes Neurais Convolucionais Profundas}
\title{License Plate Localization From Video Using Convolutional Neural Networks}

\palavraschave{Localização de Placas Veiculares, Redes Neurais Convolucionais}
\keywords{License Plate Localization, Convolutional Neural Networks}

%======Comentário da folha de rosto.
\comentario{\UTFPRdocumentodata\ apresentado ao Departamento Acadêmico de
Informática como requisito parcial para obtenção do grau de Bacharel
em Engenharia de Computação da
\ABNTinstituicaodata\ .
}

%======Local e data
\local{Curitiba}
\data{\the\year}


%=======================================================================
%Início do documento
%=======================================================================

\begin{document}
\setcounter{figure}{200}

%--------------------------------------
%Elementos Pré-Textuais
%--------------------------------------

\capa

\folhaderosto

%ficha catalográfica: deve ser impressa separadamente no verso da folha de rosto. somente para a versão final do documento (biblioteca).

\begin{dedicatoria}
	Dedico este trabalho ao meu filho Luka.
\end{dedicatoria}

\begin{agradecimentos}
	Muito obrigado.	
\end{agradecimentos}

\begin{resumo}
	%Exemplo de resumo

Localização de placas veiculares é um componente importante de sistemas de
controle e fiscalização de tráfego contemporâneos. Alguns sistemas incluem
localização como parte fundamental de um subcomponente. Em alguns casos
o desempenho do sistema de localização pode definir a viabilizada do sistema
como um todo. Neste trabalho, é proposto um método de localização aproximado
de placas veiculares em vídeo desenvolvida usando redes neurais convolucionais
aplicadas diretamente aos valores RGB dos \emph{pixels}. Na implementação aqui
exposta a placa pode ser localizada com uma margem de erro de
$ \pm 20 \times \pm 60$ \emph{pixels}.
O treinamento é feito aplicando distorções às imagens, como
ruído e variação aleatória de brilho, contraste e saturação, para produzir um
modelo robusto, mesmo sem uso de filtros ou normalização. A proposta inclui um
método eficiente de particionar a imagem de forma a reduzir consideravelmente o
número de vezes que a rede neural precisa ser aplicada. Os indicadores
\emph{precision} e \emph{recall} obtidos foram, respectivamente, 98,9\% e
96,7\% enquanto cada quadro de $480 \times 768$ é processado em 67,7 ms em
um notebook Core i5 com GPU GTX 750M.

\end{resumo}

\begin{abstract}
	%Exemplo de abstract

This is a proposal for a method that calculates vehicle's license plate
approximate
localization in video using convolutional neural networks applied directly to
the value of the RGB pixels. During the training some random distortions are
added to the images, such as adding noise and changing image's brightness,
contrast and saturation, to produce a robust model, even when used without any
filter or normalization. This proposal includes a method for efficient image
partitioning, in such a way that the number times the neural network have to be
applied is considerably reduced. Measured precision was 98.7\% and recall was
96.7\% for locating licence plates with error smaller than $60\text{px} \times
20\text{px}$.

\end{abstract}


%Listas. Obrigatório quando constar no desenvolvimento do trabalho. Consultar o documento das normas da ABNT para maiores detalhes.
\listadefiguras
\listadetabelas
\listadesiglas
%\listadesimbolos

\sumario

%--------------------------------------
%Texto
%--------------------------------------

%Incluir arquivos .tex da pasta src/
%Exemplo de capitulo

\chapter{Introdução}

Soluções de monitoração e fiscalização de veículos apresentam
uma grande demanda por sistemas de visão computacional. Seja
para contar tráfego, fiscalizar o uso das vias, monitorar
rodízios ou para cobrar pedágios, a capacidade de obter
informações a partir de imagens é de grande interesse, sendo parte integrante
de uma solução típica \cite{anagnostopoulos2008license}.  Existem
soluções propostas para parte destes problemas, no entanto há a
possibilidade da busca por estratégias com menor custo ou melhor
desempenho, à medida que avanços no campo do reconhecimento de
imagens acontecem.

Redes neurais convolucionais (CNN, convolutional neural networks)
são tipos de redes neurais biologicamente inspiradas no conceito
de campos receptivos \cite{hubel1968receptive}. Este tipo de
rede viabilizou o uso de redes neurais que são alimentadas
diretamente pelos pixels da imagem \cite{lecun1998gradient}. O
advento das Redes Neurais Convolucionais Profundas (DCNN, deep CNN)
viabilizou um grande salto na capacidade de sistemas computacionais
de classificação. O uso de mais camadas permite níveis maiores de
abstração na análise da imagem, o que resulta em maior taxa de
acerto, ao custo de maior tempo de treinamento. Este tipo de rede
neural representa o atual estado-da-arte em reconhecimento de
imagens \cite{szegedy2015going}.

Este projeto procura fazer aplicação direta de DCNN para localizar
placas de veículos em vídeos como forma de produzir um sistema
robusto, flexível, e de alto desempenho, como ilustrado na figura 1.

\section{Objetivo}
O objetivo deste trabalho é desenvolver uma abordagem para
treinamento e uso de redes neurais convolucionais para para a
construção de um sistema robusto e eficiente de localização de
placas de veículos em vídeo, e realizar uma implementação para
medição de dados de desempenho.

\section{Estrutura da Monografia}
O restante deste trabalho é organizado da seguinte forma: O
capítulo 2 discorre sobre redes neurais convolucionais em geral,
comparando-as com as redes neurais não-convolucionais. Como este
tópico é relativamente novo optou-se por fazer sua apresentação
o mais cedo possível no documento. O capítulo 3 é uma revisão da
bibliografia atual sobre detecção e localização de placas veiculares. O
capítulo 4 apresenta uma proposta de um método para aplicar redes neurais
convolucionais para localizar placas veiculares. O capítulo 5 apresenta uma
implementação do método proposto, experimentos e os seus resultados. O
capítulo 6 apresenta as conclusões deste trabalho.

 % introdução
%Exemplo de capitulo

\chapter{Redes Neurais Convolucionais}
\setcounter{figure}{100}

Rede neural artificial (ANN) é um modelo computacional inspirado na forma com
que o cérebro resolve problemas \cite{gilbert2000build}. Este modelo possui
unidades, denominadas neurônios, que possuem um valor que é calculado como uma
função do valor de outros neurônios.

\begin{figure}[!htb]
	\centering
	\includegraphics{ex_fcnn.png}
	\caption{Exemplo de rede neural totalmente conectada}
	\label{fig:ex_fcnn}
	(próprio autor).
\end{figure}

Alguns neurônios especiais, denominados “entradas” possuem um valor obtido de
fora do sistema. Estes são especiais porque são os únicos cujos valores não são
calculados. O valor de alguns dos neurônios de uma rede neural artificial podem
afetar sistemas externos à ela. Por isso são designados “saídas”. Neurônios que
não são de entrada ou de saída são chamados “ocultos”. A figura
\ref{fig:ex_fcnn} ilustra uma rede neural simples com os três tipos de
neurônios.

\section{Redes Neurais Não-Convolucionais}
As redes neurais artificiais são tipicamente organizadas em “camadas”, e a
escolha dos tipos de neurônio, juntamente com a forma com que os neurônios são
conectados, é denominada “topologia”. As redes são classificadas como
feedforward quando as conexões não formam ciclos ou recurrent quando formam. Se
uma camada está conectada a todos os neurônios da camada anterior diz-se que a
camada é totalmente conectada.
A definição da função de transferência do neurônio é uma parte importante da
topologia das redes neurais. Um tipo muito comum de neurônio é definido por:

\begin{equation} \label{eq:non-conv-layer}
	v=A \left( \left( \sum_n w_n i_n \right) + b \right)
\end{equation}

Onde v é o valor do neurônio, in é n-ésima entrada do neurônio, w é um vetor de
escalares denominado peso, e b é um escalar arbitrário, denominado bias. A é uma
função denominada função de ativação do neurônio. Essa função pode ser usada
para tornar o neurônio não-linear, como no caso da tangente hiperbólica. Os
pesos determinam a influência de cada entrada do neurônio.

A descrição de redes neurais com este tipo de neurônio e com topologia
totalmente conectada é especialmente conveniente. A camada de entrada é
representada por um vetor:

\begin{equation} \label{eq:l0}
	L0_{(4;1)}
\end{equation}

Para calcular a primeira camada intermediária é preciso aplicar
\ref{eq:non-conv-layer}. No caso de camadas totalmente conectadas cada
neurônio possui como entrada todos os
neurônios da camada anterior. Por este motivo $\sum_n w_n i_n$ pode ser
descrito como uma multiplicação de matrizes. O valor de todos os neurônios
da camada intermediária podem ser calculados com:

\begin{equation}
	L2_{(1;2)}=A_V \left( L1_{(1;2)} \cdot W2_{(5;2)} + B2_{(1;2)} \right)
\end{equation}

As matrizes $W$ e $B$ são os parâmetros que precisam ser aprendidos durante o
treinamento denominados parâmetros treináveis. O número de parâmetros de uma uma
rede neural é igual ao número de valores contidos em todas as matrizes de pesos
e bias. Quanto maior o número de parâmetros mais flexibilidade a rede neural
possui, porém mais lento é o treinamento. Se o número de parâmetros for
excessivo a rede neural perde a capacidade de generalizar e ocorre overfitting.

Como as dimensões da matriz de pesos são iguais ao número de entradas em uma
direção pelo número de saídas na outra direção, o número de parâmetros nessa
matriz é igual ao produto dos dois. Se uma camada convolucional tem 1000
entradas e 1000 saídas a matriz de pesos vai possuir 1.000.000 de parâmetros.

\section{Redes Neurais Convolucionais}
Redes neurais convolucionais operam sobre tensores, que são uma extensão de
vetores que admite um número arbitrário de dimensões. O motivo para isso é
permitir preservar a geometria da informação que está sendo processada. Uma
imagem bidimensional com dimensões $H$ e $W$ com C canais de cor vai ser
representada por um tensor de dimensões $H \times W \times C$. Uma imagem
tridimensional com 3
canais com profundidade $D$ é representada por um tensor
$D \times H \times W \times C$, e assim por
diante. Isso permite preservar o posicionamento relativo das informações durante
o processamento.

Caso a rede neural esteja operando sobre um tipo de informação que não possui o
conceito de “canal”, como séries temporais, é necessário artificialmente
criá-lo. Esse é o caso de séries temporais com “N” entradas, que serão
representadas por um tensor com dimensões $N\times1$. O motivo para isso é
uniformidade. O tensor de saída de uma camada convolucional inclui a dimensão
“canal”. Forçando tanto os tensores de entrada quanto de saída a incluírem um
canal o mesmo conceito de camada convolucional é aplicável em todas as camadas
com o mesmo número de dimensões.

A principal operação das redes neurais convolucionais é a cross-correlação:

\begin{equation}
	def: (f * g)[n] = \sum_{m=-\infty}^{\infty} f^*[m]g[m+n]
\end{equation}

Onde $f*$ é o complexo conjugado da função $f$. A cross-correlação é
equivalente à convolução de $f[n]$ com $g[-n]$. Uma camada de uma rede
neural que use qualquer uma das operações é denominada “convolucional”,
pois neste contexto as operações são intercambiáveis. A partir deste
ponto, apenas o termo “convolução” será usado.

Uma camada convolucional é definida pela a aplicação de um “kernel” sobre a
entrada dessa camada usando o operador convolução. O kernel, ou filtro é um
tensor a ser treinado. Se a entrada da camada convolucional tem dimensões
$D0 \times D1 \times D2 \times ... \times C$, onde C é o número de canais,
o filtro terá o mesmo número de dimensões, sendo que o tamanho de todas
as dimensões, exceto a última são
hiperparâmetros arbitrários. A última dimensão do filtro precisa ser igual ao
número de canais da imagem de entrada da camada convolucional, como ilustrado na
figura \ref{fig:ex_conv_2d}.


\begin{figure}[!htb]
	\centering
	\includegraphics{ex_conv_2d.png}
	\caption{Exemplo de convolução 2D}
	\label{fig:ex_conv_2d}
	Exemplo da aplicação de uma única convolução sendo aplicada à uma imagem de
	entrada $6 \times 6 \times 2$. O tamanho da última dimensão do filtro e
	da entrada precisam ser iguais, no caso 2. As dimensões $3 \times 3$ do
	filtro foram escolhidas arbitrariamente. O símbolo do círculo na imagem
	representa um produto interno. A imagem de saída foi gerada deslocando
	a seleção de pixel a pixel (stride $1 \times 1$) até cobrir toda a
	imagem de entrada. Como as bordas não foram estendidas a imagem
	resultante é menor que a de entrada (próprio autor).
\end{figure}

Camadas convolucionais também usam o conceito de bias. Para isso o tensor
resultante da convolução é somado a um escalar. Este escalar é treinável e
permite, entre outras coisas, que a rede neural gere valores não-nulos mesmo que
a entrada seja nula.

\section{Bordas}
Na imagem \ref{fig:ex_conv_2d} o tamanho do tensor de saída foi reduzido
de 6 para 4. Pode ser
desejável fazer a saída ter o mesmo tamanho da entrada. Para isso é necessário
estender as bordas da imagem de entrada com zero, de forma a ser possível
aplicar o filtro inclusive nos pixels de borda, o que faria a saída ilustrada
nesta imagem ser $6 \times 6$.

A convolução só se encaixa corretamente na definição dada na equação X quando é
usada extensão de borda com zeros.

A opção de extensão de borda é um hiperparâmetro.

\section{Stride}
Ao se aplicar o filtro de convolução no tensor de entrada pode-se movê-lo de
posição a posição, até cobrir todos os locais válidos em cada direção, como
ilustrado na figura \ref{fig:ex_conv_2d}, ou pode-se desejar aplicar a cada
"$n_0$" pixels em uma direção, “$n_1$” pixels em outra direção, e assim por diante.
Esta opção é denominada \emph{stride}.

Quando uma camada convolucional possui como entrada um tensor de dimensões
$d1\times d2 \times ... \times dj \times C$ o stride é definido como sendo
um tensor unidimensional de tamanho j.
O stride $[1,2,1]$, por exemplo, indica que na primeira dimensão o
filtro será aplicado em todas as posições válidas, na segunda dimensão será
aplicado a cada 2 pixels e na terceira será aplicada novamente em todas as
posições válidas. Stride maior que 1 causa redução no tamanho do tensor de
saída.

Se um stride $2 \times 2$ fosse aplicado na convolução da figura
\ref{fig:ex_conv_2d} a imagem de saída seria $2 \times 2$.

\section{Profundidade do Filtro}
A figura \ref{fig:ex_conv_2d} mostrou um único filtro sendo aplicado ao
tensor de entrada. No
entanto, é possível aplicar um número arbitrário deles, sendo que cada filtro
produz um tensor de saída. Cada um destes filtros pode ser treinado para
reconhecer um feature diferente.

Se o tensor de entrada possui dimensões $d0 \times d1 \times ... \times C$
e deseja-se aplicar sobre este
tensor $N$ convoluções é possível representar isso com um único tensor
com dimensões
$N \times d0 \times d1 \times ... \times C$. Cada um dos filtros vai gerar
um tensor de saída, em um total de $N$. O
número de filtros é denominado “profundidade do filtro”. Todas as saídas podem
ser representadas em um único tensor com uma dimensão adicional. Usa-se a última
dimensão do tensor de saída para este fim. O motivo para isso é que cada uma
dessas saídas efetivamente se torna um “canal” de saída desta rede neural. No
exemplo da figura \ref{fig:ex_conv_2d}, se fossem aplicados 3 filtros a saída
seria um tensor $4 \times 4 \times 3$. Para isso o tensor que define o filtro
convolucional passaria a ser um
filtro de profundidade 3 descrito por um tensor de dimensões
$3 \times 4 \times 4 \times 2$.

\section{Processamento em Lotes}
Existem alguns cuidados especiais que devem ser tomados durante o treinamento de
redes neurais. Tomando o caso do treinamento supervisionado como exemplo, o
otimizador ajusta os parâmetros treináveis da rede neural para tentar reduzir o
erro. Para verificar se a alteração teve sucesso a rede neural é alimentada com
dados etiquetados, e o valor de saída da rede neural é comparado com os dados
esperados. É conveniente fornecer vários exemplos etiquetados, não apenas um, e
usar a média do erro para alimentar o otimizador. A idéia é que o valor passado
seja o mais representativo possível.

O número de imagens fornecidas é denominado “tamanho do batch”. Se uma rede
neural trabalha com dados de dimensões $d0 \times d1 \times ... \times C$
pode-se agrupar “B” exemplos em
um único tensor, com dimensões $B \times d0 \times d1 \times ... \times C$.
As camadas convolucionais tratam cada
uma das imagens separadamente, conforme já foi descrito, e emite na sua saída um
único tensor que também usa uma dimensão adicional para agrupar as diferentes
imagens.

O valor de B não é um hiperparâmetro, mas sim um parâmetro de treinamento.

\section{Pooling}
Após a aplicação de uma camada convolucional pode-se aplicar uma camada de
\emph{pooling}, que é uma forma de subamostragem. A figura
\ref{fig:ex_maxpool} ilustra uma operação de maxpool, que é um tipo de
pooling.

\begin{figure}[!htb]
	\centering
	\includegraphics{ex_maxpool.png}
	\caption{Exemplo de \emph{maxpool} $2 \times 2$}
	\label{fig:ex_maxpool}
	Ilustração de uma operação maxpool $2 \times 2$ aplicado usando stride
	$2 \times 2$. O filtro move de dois em dois pixels e possui
	tamanho 2 em cada dimensão (próprio autor).
\end{figure}

Esta operação possui como parâmetros o tamanho do filtro, o stride e a opção de
borda. A operação a ser aplicada no filtro pode ser $max$ ou $avg$ (média),
que definem respectivamente os filtros \emph{maxpool} e \emph{avgpool}.

Como as operações de pooling são normalmente feitas com stride maior que 1 elas
acabam reduzindo consideravelmente o tamanho do tensor de saída. No caso de um
maxpool $2 \times 2$, por exemplo, o tensor resultante vai ter 25\% do
número de valores do tensor de entrada.

\section{ReLu}
Assim como em redes neurais totalmente conectadas, pode-se aplicar uma função de
ativação para aumentar a não-linearidade. Pode-se usar as funções tangente
hiperbólica, sigmóide ou outras tipicamente usadas em redes neurais
não-convolucionais. No entanto a função ReLu, ou linear retificada, resulta em
treinamento substancialmente mais rápido enquanto mantém a capacidade de
generalização da rede neural treinada. A função ReLu é definida por:


\begin{equation}
	ReLu(x) = max(0,x)
\end{equation}

\section{Últimas Camadas}
Após as camadas convolucionais terem sido aplicadas é necessário usar um
classificador, mecanismo de regressão ou outro sistema que gere o tipo de saída
desejada para a rede neural. Uma das possíveis formas de realizar esta função é
usar uma ou duas camadas totalmente conectadas, como ilustrado na figura
\ref{fig:ex_cnn}. No exemplo a função de ativação da penúltima camada é
ReLu e a última camada é linear.

\section{Rede Neural Convolucional Completa}
Para alguns casos simples pode-se construir uma rede neural convolucional
conectando-se uma camada convolucional à uma maxpool e uma ReLu. Este conjunto
pode ser repetido algumas vezes até que o número de saídas da camada
convolucional seja baixo o suficiente para que seja passado por duas camadas
totalmente conectadas. Se houver interesse em não reduzir o tamanho total do
tensor pode-se omitir a camada maxpool. Um exemplo dessa topologia é a figura
\ref{fig:ex_cnn}.

A topologia da rede neural e o conjunto de todos os hiperparâmetros definem o
número de parâmetros treináveis que a rede neural vai possuir e quantas
operações são necessárias para aplicar a rede neural em um (ou um lote de)
amostras. Uma rede neural mais larga, com um maior número de filtros por
camadas, possui a capacidade de aprender mais features. Redes neurais mais
profundas possuem capacidade de abstração maior, sendo capazes de inferir a
partir dos dados de entrada conceitos mais complexos.

\begin{figure}[!htb]
	\centering
	\includegraphics{ex_cnn.png}
	\caption{Exemplo de rede neural convolucional completa}
	\label{fig:ex_cnn}
	Esta rede neural possui nove camadas para
	classificação de séries temporais em uma de 4 classes. Destaca-se os
	parâmetros treináveis (próprio autor).
\end{figure}

\section{Uso em Processamento de Imagens}
O uso de redes neurais convolucionais é uma alternativa a diversos métodos já
existentes de detecção de objetos em imagens. Aqui serão mostradas algumas das
alternativas usuais, para que sejam contrastadas com a classificação baseada em
redes neurais convolucionais.

\subsection{Redes Neurais Não-Convolucionais}
Quando uma imagem vai ser processada por uma rede não-convolucional ela precisa
ser convertida para a forma plana, em um vetor com dimensões
$(1; H \cdot W \cdot C)$.

O uso de camadas totalmente conectadas é proibitivo para classificação de
imagens desta maneira. Se uma rede neural for usada para processar imagens de 1
megapixel, ou 720x1280, e a primeira camada possuir um número de neurônios igual
ao número de neurônios da camada de entrada, a matriz de pesos teria
$(720 \cdot 1280 \cdot 3)^2 \approx 7.6 \cdot 10^{12}$ parâmetros. Se for
representado por números ponto-flutuante de 32 bits isso ocuparia mais de
30 TiB. O mesmo processo em uma imagem 480px por 640px geraria quase 850
bilhões de parâmetros só na matriz de pesos.

Outras topologias mais esparsas podem ser construídas, mas este tipo de rede
neural possui outros problemas que as tornam inviáveis para processamento direto
de imagens. Um exemplo é o fato delas não serem invariantes ao deslocamento. Se
a rede neural aprende a reconhecer um feature em um local da imagem não vai
conseguir reaproveitar essa capacidade em outras posições. Portanto teria que
aprender a mesma feature em cada possível posição onde ela possa aparecer, e
fazer isso para todas as features.

Como a primeira operação feita foi converter a imagem para um formato “plano”, a
geometria da imagem foi destruída. Não é mais possível determinar quando dois
pixels estão próximos, e essa é uma informação muito importante sobre a imagem.

\subsection{Redes Neurais Convolucionais em Imagens}
As redes neurais convolucionais resolvem todos esses problemas simultaneamente.
O primeiro ponto é a preservação da geometria da imagem. Se uma imagem de 50
pixels de altura por 100 pixels de largura contendo 3 canais de cor for usado na
entrada de uma rede neural convolucional, ela precisa ser representada por um
tensor apropriado, como $32 \times 40 \times 120 \times 3$. Este tensor
permite alimentar a rede neural com 32 imagens.

Para aplicar uma camada convolucional à imagem de entrada escolhe-se o tamanho
da convolução, o número de filtros e o modo de operação nas bordas. A figura
\ref{fig:ex_cnn_img} ilustra um filtro 33 sendo aplicado à uma imagem RGB.

\begin{figure}[!htb]
	\centering
	\includegraphics{ex_cnn_img.png}
	\caption{Convolução sendo aplicada em uma imagem RGB}
	\label{fig:ex_cnn_img}
	Como a imagem possui 3 canais o filtro será 3x3x3. Um segmento da imagem
	original com o mesmo tamanho do filtro, 3x3x3 é extraído da imagem, e um
	produto interno é realizado entre os dois, resultando em um escalar. Este
	escalar é armazenado no tensor de saída nas coordenadas corretas. A
	ilustração só mostra uma imagem de entrada e só um filtro e, portanto, uma
	imagem de saída (próprio autor).
\end{figure}

Para que o filtro tenha profundidade 64, por exemplo, o tensor que define o
filtro será $64 \times 3 \times 3 \times 3$ e o tensor resultante da
convolução será $32 \times 40 \times 120 \times 64$,
considerando que seja usado padding nas bordas.

Para finalizar a camada convolucional adiciona-se um bias. Para isso usa-se um
escalar para cada uma das 64 imagens resultantes. O tensor que define os bias é
um tensor unidimensional com tamanho 64.

Camadas maxpool e ReLu podem ser adicionadas para reduzir as dimensões da imagem
ou aumentar a sua não-linearidade. A figura \ref{fig:ex_full_cnn_img}
ilustra o uso das três camadas em sequência.

\begin{figure}[!htb]
	\centering
	\includegraphics{ex_full_cnn_img.png}
	\caption{Exemplo de rede neural convolucional completa aplicada à uma imagem
	RGB}
	\label{fig:ex_full_cnn_img}
	Exemplo de uma sequência de operações feitas usando uma rede neural
	convolucional incluindo 64 convoluções 3x3, maxpool e ReLu aplicadas em uma
	batch de 32 imagens $40 \times 120 times 3$. Os parâmetros treináveis
	estão destacados (próprio autor).
\end{figure}

A imagem é tratada por camadas sucessivas de convolução, maxpool e ReLu até que
o tamanho total do tensor seja pequeno o suficiente para poder ser processado
por duas camadas totalmente conectadas. 

\subsection{Uso de Cores}
Em alguns sistemas de visão computacional costuma-se usar imagens em
tons de cinza. No caso de redes convolucionais o mais comum é usar
imagens coloridas.

O primeiro motivo para isso é que as imagens e vídeos são normalmente
captadas com cores, então colorir vai requere processar a imagem para
que a conversao seja feita. 

Pode-se demonstrar todos os métodos de conversão lineares de RGB para escala de
cinzas, como:

\begin{equation} \label{eq:rgb2gray}
	G=0.299R + 0,587G + 0.114B
\end{equation}

Podem ser representados como uma convolução $1 \times 1$. Visto que as
convoluções são implementadas de forma bastante eficientes nas bibliotecas
de redes neurais convolucionais, algumas vezes até usando aceleração por
hardware, então não necessariamente haverá economia durante a execução
durante a execução da rede neural, pois, no pior caso, ela mesma pode fazer
essa conversão de forma eficiente, sendo que os coeficientes associados
à cada canal, que no exemplo da equação \ref{eq:rgb2gray} são fixos, podem
treinados pelo otimizador, potencialmente obtendo uma conversão mais apropriada
para a rede neural.

Outro ponto é que os sistemas os classificadores de imagem usam cores
\cite{szegedy2015going} \cite{hasanpour2016lets}, portanto nos casos onde
existe \emph{budget} computacional suficiente para se usar um número razoável
de convoluções na primeira camada, a informação da cor pode ser vantajosa
para para minimizar os erros da rede neural.


\subsection{Reconhecimento de Objetos}
Existem várias maneiras de se usar redes neurais para reconhecer objetos. Duas
das opções envolvem configurar a neural como classificador e como sistema de
regressão.

Para usar a rede como classificador pode-se incluir na última camada da rede
neural um sistema softmax, também conhecido como exponencial normalizada. Isso
permite que a rede neural seja treinada e, após o treinamento, estime a
probabilidade da sua entrada pertencer a cada uma de N classes. A rede neural é
implementada usando $N$ saídas, de forma que cada saída é treinada para
representar a probabilidade da entrada ser classificada em em uma das classes.
Uma saída \emph{softmax} realiza uma normalização das saídas da rede neural,
usando a equação:

\begin{equation}
	softmax[i] = \frac
		{exp\left( \widehat{out}[i] \right)}
		{\sum_j exp\left( \widehat{out}[j] \right)}
\end{equation}

Para identificar a presença ou não de uma placa de um veículo, duas classes
seriam definidas: "placa" e "não-placa", resultando em uma rede neural como
mostrada na figura \ref{fig:ex_classif_logist}.

\begin{figure}[!htb]
	\centering
	\includegraphics{ex_classif_logist.png}
	\caption{Exemplo de Classificador Logístico}
	\label{fig:ex_classif_logist}
	Este exemplo de rede neural possui uma imagem como entrada e possui duas
	saídas, uma informando a probabilidade da imagem conter uma placa e a
	outra indicando a probabilidade de não conter (próprio autor).
\end{figure}

Este modelo pode ser treinado usando modo supervisionado, onde imagens
pré-classificadas são fornecidas. A função de perda a ser minimizada pelo
otimizador é a função cross entropy:

\begin{equation}
	\widehat{H} = - \sum_i \widehat{y_i} log(y_i)
\end{equation}

Onde $y$ é um tensor de dimensão $2$ que indica as probabilidades manualmente
marcadas de uma imagem pertencer a cada classe (no caso, a probabilidade ser uma
placa de carro e a probabilidade de não ser) e $y$ é um tensor com o mesmo número
de dimensões que indica o valor estimado pela rede neural. $H$ é o valor estimado
de cross entropy.

No nosso caso temos total certeza de que uma imagem é ou não uma placa, então
serão usados somente os valores $0$ e $1$ para $y$.

As redes neurais também podem ser usadas para fazer regressão, ou seja, modelar
uma função. A forma mais óbvia de implementação é a regressão L2. Neste tipo de
função o otimizador vai minimizar o erro quadrático:

\begin{equation}
	E=\left( y - \widehat{y} \right)^2
\end{equation}

A rede neural pode ser usada para modelar a função $P$, definida por:

\begin{equation}
	P(img) = \begin{cases}
		1 \text{, se img contém uma placa} \\
		0 \text{, se img não contém uma placa}
	\end{cases}
\end{equation}

Exatamente essa definição não seria ideal. Uma definição alternativa que possa
representar uma placa parcialmente visível na imagem usando um número entre 0 e
$1$ vai ser necessária para evitar a criação de ruído no valor da perda que está
sendo minimizado.

\subsection{Localização de Objetos}
Até agora foi mostrado como usar redes neurais convolucionais para detectar um
objeto em uma imagem. Isso envolve apenas determinar se uma imagem possui ou não
o objeto de interesse. Localização, ao contrário, requer a determinação das
coordenadas do objeto.

Um sistema de localização pode ser construído a partir de um sistema de
detecção. A abordagem mais simples para isso envolve construir um sistema capaz
de detectar o objeto de interesse, no caso uma placa de trânsito, e aplicar este
detector à imagem usando um operador de cross-correlação em duas dimensões. Se o
detector for construído para produzir “1” quando detectar o objeto e “0” então
os pontos de máxima são candidatos a centros dos objetos.

Como em uma cross-correlação, assim como na convolução, cada número de um dos
operandos é multiplicado por todos os números contidos no outro operando, como
demonstrado visualmente na figura \ref{fig:ex_sliding_stride1}. A propriedade
é extensível para qualquer número de dimensões. O número de somas é igual ao
número de multiplicações menos 1.


\begin{figure}[!htb]
	\centering
	\includegraphics{ex_sliding_stride1.png}
	\caption{Multiplicações resultantes do uso de janela deslisante}
	\label{fig:ex_sliding_stride1}
	Demonstração visual de que cada termo de cada operando em uma operação de
	cross-correlação é multiplicado exatamente uma vez por cada um dos termos do
	outro operando, resultando em um número de multiplicações igual ao produto
	do número de termos em cada operando. (próprio autor)
\end{figure}

A consequência disso é que quando se calcula a cross-correlação entre duas
imagens o número total de multiplicações é igual ao produto do número de pixels
em cada imagem.
 % redes neurais convolucionais
%Exemplo de capitulo

\chapter{Trabalhos Relacionados}

Este \'e o segundo cap\'itulo, a fundamenta\c{c}\~ao te\'orica.
 % trabalhos relacionados
%Exemplo de capitulo

\chapter{Localização de Placas Veiculares usando Redes Neurais Convolucionais}

O método proposto para a extração de placas envolve uma série de etapas,
conforme ilustrado na figura \ref{fig:etapas_metodo_proposto}.

\begin{figure}[!htb]
	\centering
	\includegraphics{etapas_metodo_proposto.png}
	\caption{Etapas do método proposto para localização de placas veiculares}
	\label{fig:etapas_metodo_proposto}
	(próprio autor).
\end{figure}

O processo “extração” é responsável por obter as frames do vídeo. Uma frame é
equivalente a uma imagem 2D, e pode ser representada por tensores de dimensões
HWC (largura, altura e canais de cor). Todos os frames de um mesmo vídeo geram
tensores do mesmo tamanho.

O processo “segmentação” toma como entrada uma frame do vídeo e gera múltiplas
sub-imagens aqui denominados “segmentos”, gerados através de um recorte
retangular de pixels contíguos. Os segmentos têm dimensões compatíveis com a
rede neural que será usada no passo seguinte.

A etapa denominada “rede neural” é, mais especificamente, uma rede neural
artificial convolucional profunda treinada para modelar uma função escalar cujo
valor, denominado “escore”, é determinado pela presença e posição, ou não
presença, de uma placa veicular.

A etapa “binarização” estima, a partir do resultado da rede neural, se um
segmento possui uma placa de trânsito.

Este processo envolve algumas decisões que são parte fundamental da atual
proposta. Elas dizem respeito:

\begin{itemize}
\item ao tamanho dos segmentos de imagem a serem recortados e, por
	consequência, ao tamanho da entrada da rede neural;
\item a escolha do stride dos segmentos da rede neural
\item a escolha da função que a rede neural está modelando
\item a escolha da arquitetura e dos hiperparâmetros da rede neural;
\end{itemize}

\section{Tamanho dos Segmentos}
O tamanho dos segmentos definem as dimensões do tensor da rede neural, portanto
o tamanho do seu “campo visual”.

\begin{figure}[!htb]
	\centering
	\includegraphics{ex_tres_segmentos.png}
	\caption{Exemplo de três tamanhos de segmentos}
	\label{fig:ex_tres_segmentos}
	Em (a) o segmento possui um grande campo visual, porém tem baixa precisão.
	Em (b) há um campo visual muito pequeno para permitir que a rede neural
	opere corretamente. O terceiro é apropriado, pois é relativamente pequeno e
	inclui a placa inteira e vários pixels adicionais, permitindo que a rede
	neural aprenda o contraste entre placa e o fundo (autoria própria).
\end{figure}

Quanto maior o campo visual da rede neural mais dados ela vai ter para
processar, porém menor será a precisão. Se o campo visual for muito pequeno,
como na figura \ref{fig:ex_tres_segmentos}b a precisão seria maximizada,
mas a área não é suficiente para a rede neural operar corretamente. Se
o campo visual for muito grande, como na figura \ref{fig:ex_tres_segmentos}a,
a rede neural terá mais características para usar, mas a precisão da
localização será muito prejudicada.

De forma geral, as dimensões dos segmentos devem ser minimizados, mas deve-se
garantir que a placa caiba inteira caiba no seu interior, com um pouco de
sobra. As figuras \ref{fig:ex_tres_segmentos}c e \ref{fig:ex_placa} mostram
um bom tamanho. A região adicional fora da placa vai servir para a rede
neural aprender que existe um contraste de features entre o
interior e o exterior da placa. Este contraste vai ser usado não só para
identificar quando existe uma placa, mas também para excluir muitos casos de
falsos positivos, como banners e outros textos que podem ser encontrados em
veículos.

\begin{figure}[!htb]
	\centering
	\includegraphics{ex_placa.png}
	\caption{Exemplo de segmento com bom tamanho}
	\label{fig:ex_placa}
	(próprio autor).
\end{figure}

\section{Strides dos Segmentos}
A abordagem que está sendo proposta envolve usar um detector para construir um
localizador. Uma implementação ingênua desta abordagem seria treinar o detector
para detectar placas de trânsito que estejam centradas na sua entrada e aplicar
este detector como descrito na sessão \ref{sec:localiz_objetos}, usando
stride $1 \times 1$.  Isso iria requerer aplicar a rede neural centrada em
cada pixel da imagem. Se for feita a aplicação de um detector com campo
visual $D_1 \times D_2$ em uma frame com dimensões $F_1 \times F_2$ com stride
$1 \times 1$ sem estender as bordas pode-se demonstrar que o detector terá
que ser aplicado $N$ vezes, onde:

\begin{equation}
	N = (F_1 - D_1 + 1) \cdot (F_2 - D_2 + 1)
\end{equation}

Para um detector com dimensões HW de $40 \times 120$ em uma imagem
$480 \times 768$ seriam gerados 286.209 segmentos, e cada um deles precisaria
ser aplicado pelo detector, o que é proibitivo.

Para resolver este problema propõe-se o uso de um stride de
$50\% \times 50\%$ do tamanho do segmento. Isso significa que duas amostras
consecutivas em linha tem os seus centros a uma distância igual a sua
largura sobre dois, gerando segmentos como os ilustrados na figura
\ref{fig:ex_3_segmentos}.

\begin{figure}[!htb]
	\centering
	\includegraphics{ex_3_segmentos.png}
	\caption{Ilustração de 3 segmentos sucessivos}
	\label{fig:ex_3_segmentos}
	Observar que o centro de um segmento está contido no perímetro dos
	segmentos vizinhos (próprio autor).
\end{figure}

Uma propriedade dessa escolha é que o centro de um segmento de imagem está
contido no perímetro de todos os segmentos vizinhos. Esta propriedade é
justamente o objetivo do stride de 50\%, conforme será descrito na sessão
\ref{ses:funcao_a_modelar}.

O número de segmentos na direção da altura será $T_H$, e na largura será
$T_W$, sendo:

\begin{equation}
	T_H=\left\lfloor \frac{2F_1}{D_1-1} \right\rfloor \\
\end{equation}

\begin{equation}
	T_W=\left\lfloor \frac{2F_2}{D_2-1} \right\rfloor \\
\end{equation}

E o  número de segmentos a serem classificados será:

\begin{equation}
	N=\left\lfloor \frac{2F_1}{D_1-1} \right\rfloor \cdot
		\left\lfloor \frac{2F_2}{D_2-1} \right\rfloor
\end{equation}

Tomando o mesmo caso calculado anteriormente (detector $40 \times 120$ em
frames $480 \times 768$ sem extensão de borda) com a estratégia proposta
de segmentação o número de segmentos a serem classificados cai de 286.209
para $24 \cdot 12=288$.

\section{Função a ser Modelada} \label{ses:funcao_a_modelar}

A rede neural vai ser treinada para modelar o valor de uma função $P$ que
leva um tensor que representa  um segmento da frame à um real no intervalo
$[0;1]$:

\begin{equation}
	P:S \to [0;1] \in \mathbb{R} 
\end{equation}

O valor dessa função é resultante da presença de uma placa veicular. Devido ao
fato de que a segmentação será feita com stride maior que 1 não há garantia de
que a placa estará centralizada, e ainda assim a função precisa identificar que
ela está presente.

A função proposta foi construída baseado na forma com que a placa veicular
passa de um segmento para o segmento vizinho. A função deve possuir as
seguintes características:

\begin{itemize}
\item a função é contínua quando uma placa é movida de forma contínua na
	entrada por qualquer caminho;
\item quando a placa veicular está no centro de um segmento a função de
	saída deve ser 1, e quando está no centro do segmento vizinho deve ser 0;
\item um, e apenas um segmento deve possuir valor 1 como consequência da
	presença da placa.
\end{itemize}

A figura \ref{fig:placa_movida_entre_segmentos} ilustra os pontos principais
quando uma placa é movida horizontalmente entre dois segmentos.

\begin{figure}[!htb]
	\centering
	\includegraphics{placa_movida_entre_segmentos.png}
	\caption{Placa veicular sendo movida entre dois segmentos}
	\label{fig:placa_movida_entre_segmentos}
	Ilustração do dos pontos críticos para a definição da função a ser modelada
	pela rede neural, mostrando dois segmentos vizinhos, um pintado de verde e
	o outro amarelo com borda pontilhada. Em (a) a placa está centrada no
	segmento da esquerda e no perímetro do segmento da direita.
	Em (b) o centro da placa está no meio caminho entre os centros dos dois
	segmentos. Em (c) o centro da placa está no centro do segmento da direita e
	perímetro do segmento da esquerda.
\end{figure}

A figura \ref{fig:func_a_modelar_2_seg} mostra o valor da função para
dois segmentos vizinhos a medida que a placa é movida. Pode-se observar apenas
um dos segmentos possui valor 1 por vez, exceto em um ponto. Também observa-se
que quando a placa está centrada em um segmento a função possui valor zero
no outro segmento.

\begin{figure}[!htb]
	\centering
	\includegraphics{func_a_modelar_2_seg.png}
	\caption{Valor da função em dois segmentos a medida que a placa se move
	entre eles}
	\label{fig:func_a_modelar_2_seg}
	Observa-se que valor da saída da função para dois segmentos a medida que a
	placa é movida do centro do segmento 1 (o da esquerda) para o segmento 2 (o
	da direita) (próprio autor).
\end{figure}

\begin{figure}[!htb]
	\centering
	\includegraphics{func_a_modelar_1_seg.png}
	\caption{Valor da função a modelar, de $\infty$ a $+\infty$}
	\label{fig:func_a_modelar_1_seg}
	Observa-se que o valor da variável dependente a em medida que a placa
	é deslocada desde $-\infty$ até $+\infty$ da esquerda para a direita,
	enquanto a mesma está centrada na altura. A abscissa é a distância
	normalizada entre os centros do segmento e da placa.
\end{figure}

Na figura \ref{fig:func_a_modelar_1_seg} observa-se o valor de uma função
quando a placa está alinhada na altura e é deslocada horizontalmente. Como
a abscissa é a distância normalizada entre os centros obtém-se o valor 0,5
quando o centro da placa está a uma distância igual a metade do tamanho do
segmento para a direita, como ilustrado na figura
\ref{fig:placa_movida_entre_segmentos}c.

No caso de uma placa alinhada na direção $y$ (distância em $y$ é 0), e sendo
deslocada apenas em $x$, variando a distância normalizada $\Delta x$, a função
proposta é:

\begin{equation}
	f_x(\Delta x) = \begin{cases}
		1 \text{, se } |\Delta x| \leq 1/4
		\\
		\frac{1/2-|\Delta x|}{1/2-1/4} \text{, se } 1/4<|\Delta x|<1/2
		\\
		0 \text{, se } |\Delta x| \geq 1/2
	\end{cases}
\end{equation}

Da maneira semelhante, quando a distância em x é zero e a distância 
normalizada $\Delta y$ é livre:

\begin{equation}
	f_y(\Delta x) = \begin{cases}
		1 \text{, se } |\Delta y| \leq 1/4
		\\
		\frac{1/2-|\Delta y|}{1/2-1/4} \text{, se } 1/4<|\Delta y|<1/2
		\\
		0 \text{, se } |\Delta y| \geq 1/2
	\end{cases}
\end{equation}

Quando usado com as direções x e y livres a função a ser estimada pela rede
neural é:

\begin{equation}
	f=f_x \cdot f_y
\end{equation}

O gráfico \emph{3D} desta função está representado na figura
\ref{fig:func_a_modelar_3d}, mostrando o
efeito simultâneo dos eixos $x$ e $y$.  Observa-se que quando a placa
está com o seu centro a uma distância normalizada inferior 1/4 em $x$ e em $y$
simultaneamente, a função vai produzir o valor 1. O gráfico não possui
descontinuidades. Quando a distância normalizada supera 1/2 em qualquer
direção o valor da função é 0.

\begin{figure}[!htb]
	\centering
	\includegraphics{func_a_modelar_3d.png}
	\caption{Gráfico 3D da função à modelar}
	\label{fig:func_a_modelar_3d}
	Plot da função que a rede neural deve modelar. As abscissas representam a
	distância normalizada entre o centro do segmento e da placa em cada direção
	(próprio autor).
\end{figure}


\section{Arquitetura da Rede Neural}

A arquitetura da rede neural, incluindo seus hiperparâmetros, deve ser
escolhida de forma a balancear desempenho de classificação com tempo de
propagação. Se a rede neural requerer muitas operações pode aumentar o
desempenho de classificação, mas o tempo de processamento vai aumentar.
Idealmente a rede neural deve ser restrita a um tamanho que permita que todos
os segmentos da imagem sejam classificados de forma que o sistema como um todo
entregue a taxa de frames desejada para a resolução do vídeo necessária e
hardware disponíveis onde a solução vai ser implantada.

Na implementação da rede neural convolucional existem otimizações que podem ser
feitas para reduzir o número de operações enquanto mantendo ou reduzindo pouco
a capacidade da rede neural de executar a operação para a qual vai ser
treinada. Estas otimizações são um campo ativo de pesquisa, sendo relevantes os
papers produzidos como resultado da competição \emph{ImageNet Large Scale
Visual Recognition Competition (ILSVRC)}. Os \emph{papers}
\cite{szegedy2015going} e \cite{szegedy2015rethinking},
por exemplo, detalham várias substituições que podem ser feitas para este fim.


 % localização de placas veiculares usando cnn
%Exemplo de capitulo

\chapter{Implementação e Experimentos}

Neste capítulo serão apresentada uma implementação do método descrito
no capítulo \ref{ses:metodo}, bem como os resultados das medições de
desempenho.

Também será apresentado um breve histórico mostrando algumas abordagens que
foram tentadas antes da implementação final ter sido obtida, mostrando por que
elas falharam.

\section{Arquitetura Global}
Desde as primeiras tentativas de implementação algumas características do
software permaneceram sem alterações. Isso inclui o uso de um detector para
construir um localizador e a lista de módulos de software. Estas
características são as listadas nessa seção.

Redes neurais em geral requerem grande quantidade de exemplos de treinamento
para evitar \emph{overfitting} \cite{hawkins2004problem}. É crucial para o
sucesso de uma implementação acesso a dados ou a capacidade de produzi-los.

\begin{figure}[!htb]
	\centering
	\includegraphics{cap5_arquit_global.png}
	\caption{Arquitetura Global da Implementação}
	\label{fig:cap5_arquit_global}
	Diagrama ilustrando os quatro componentes de software em verde, os vídeos
	de entrada em azul, as informações intermediárias em vermelho e a saída do
	sistema em amarelo (próprio autor).
\end{figure}

Em todas variações tentadas de implementação optou-se por gerar os dados de
treinamento. Para tal, dois softwares foram desenvolvidos: um software de
marcação manual de placas em vídeo e um software para geração desses dados.
Usando os dois softwares é possível gerar exemplos rotulados em quantidade
suficiente no formato requerido pela rede neural. O treinamento e teste da rede
neural requereu o desenvolvimento de outros dois outros softwares. O primeiro
treina a rede neural, o segundo reproduz o vídeo enquanto aplica a rede neural
treinada, mostrando as placas identificadas. Isso resulta em quatro módulos de
software principais. O relacionamento entre eles é mostrado na figura
\ref{fig:cap5_arquit_global}.

A implementação de cada um destes módulos mudou durante a história do projeto,
porém as suas funções básicas permaneceram as mesmas.

\section{Abordagens Anteriores}
Durante o desenvolvimento do projeto algumas tentativas mal sucedidas foram
feitas antes da versão final ter sido completada. O software chegou a ser
implementado por completo três vezes, sendo que só produziu resultados
satisfatórios na última.

A ideia inicial para este projeto era o uso de redes neurais não-convolucionais
aplicadas diretamente aos pixels das imagens. Para tal foi escolhida a
biblioteca \emph{neuroph}, devido à familiaridade e ao uso da linguagem Java. A
solução foi totalmente implementada conforme arquitetura ilustrada na figura
\ref{fig:cap5_arquit_global}.

Essa implementação usava imagens em tons de cinza e partições com dimensões HW
$32 \times 100$ aplicadas em uma rede neural totalmente conectada com 3200
entradas e uma saída. O particionamento era feito usando \emph{stride} de
100\%, ou seja, duas partições vizinhas tinham o perímetro de um dos seus
lados em comum.

Esta biblioteca foi logo descartada porque o tempo de treinamento era muito
longo, impedindo a busca eficiente de configuração das camadas ocultas que
gerasse o resultado desejado. Testes com topologias mais complexas, com maior
largura e profundidade nas camadas intermediárias, chegavam a passar de uma
semana de execução quando rodadas em um notebook \emph{Core i7}. Não houve
nenhuma configuração encontrada com desempenho de classificação aceitável.

Acreditando que seria possível resolver o problema encontrando os
hiperparâmetros corretos da rede neural, e sabendo que isso requeriria
múltiplos experimentos com topologias diferentes, adotou-se a biblioteca
\emph{encog},
devido ao melhor desempenho de treinamento. Essa adaptação, requereu reescrever
boa parte código. Apesar do treinamento rodar quase a uma taxa de imagens 10
vezes maior que a biblioteca anterior, também não foi encontrada topologia
com com desempenho aceitável. A figura
\ref{fig:cap5_trein_redes_nao_conv} mostra a
evolução do treinamento para algumas sessões. Entre as listadas a que teve
melhor desempenho usava 3200 neurônios na camada de entrada, então 172, 137,
109, 87, 69, 55, 44, 35 e 28 neurônios nas camadas ocultas, terminando
com uma saída.

\begin{figure}[!htb]
	\centering
	\includegraphics{cap5_trein_redes_nao_conv.png}
	\caption{Erro de treinamento de redes neurais não-convolucionais}
	\label{fig:cap5_trein_redes_nao_conv}
	A ordenada refere-se ao erro L2 calculado sem a raiz quadrada e suavizado
	usando \emph{exponential smoothing}. A legenda refere-se ao número de
	neurônios em cada camada totalmente conectada (próprio autor).
\end{figure}

Quando o \emph{player} era usado com as redes neurais resultantes
desse treinamento o
resultado era muito ruim, com quantidade muito grande de falsos negativos.
Eventualmente descobriu-se que a forma com que os exemplos de imagens que não
continham placa estava sendo coletada possuía uma grande quantidade de imagens
muito parecidas. Aparentemente a rede neural não estava conseguindo generalizar
as características de uma placa veicular e, ao invés disso, encontrou
parâmetros para as camadas totalmente conectadas que identificavam corretamente
alguns destes exemplos, o que acabava reduzindo o erro de treinamento até certo
ponto. Este problema só foi resolvido depois das redes neurais
totalmente conectadas terem sido abandonadas.

O esquema de particionamento das imagens foi mudado de 100\% para 50\%, que é o
valor usado agora. A função que a rede neural modela foi modificada
várias vezes, mas o resultado final, conforme visto no \emph{player},
continuava ruim.

Após alguns meses a abordagem foi abandonada. Foi feita uma pesquisa sobre o
assunto descobriu-se a existência de redes neurais convolucionais. Infelizmente
a biblioteca \emph{encog} e a \emph{neuroph} não suportam este tipo de
topologia. Também não foram encontradas bibliotecas com grande base de
usuários em Java.

Neste ponto foi escolhida a biblioteca recém-lançada \emph{TensorFlow}.
Inicialmente a
preparação dos dados de treinamento continuou sendo feita pelo código em Java
que já havia sido escrito, enquanto o código de treinamento e execução foram
reescritos em Python, que era a única linguagem suportada quando essa
migração foi realizada.

O uso de duas linguagens de programação estava eliminando várias oportunidades
de reuso de código. Eventualmente os softwares de marcação e preparação de
dados de treinamento foi reescrito em Python.

Desde os primeiros experimentos a abordagem usando redes neurais foi muito bem
sucedida.

\section{Escolha de Tecnologias}
A versão final da DCNN usa o \emph{framework} \emph{TensorFlow} do Google.
No \emph{TensorFlow} usa-se a linguagem Python para descrever um ou mais
grafos onde
os nós são operações e as arestas são tensores. Uma vez que o grafo esteja
totalmente definido podem-se fornecer dados e solicitar o cálculo de qualquer
quantidade de nós. A execução em si ocorre em um \emph{runtime} de alto
desempenho escrito em C++ e CUDA, que pode ser distribuído entre múltiplas
máquinas e pode rodar em CPUs e GPUs.

A tecnologia foi escolhida por ser \emph{open source}, e por ser a ferramenta
utilizada pelo Google. Esta biblioteca é fácil de usar para pequenos projetos,
como este, e poderosa para poder escalar para sistemas com múltiplos
computadores usando GPUs de alto desempenho. O \emph{TensorFlow} permite usar
Python 2 ou Python 3, sendo que a versão 3 foi escolhida para o projeto.

A versão 3 do OpenCV, bem como seu \emph{wrapper} Python foram adotados para
leitura e exibição do vídeo, e para algumas operações de manipulação de
imagens e para a construção da interface com usuário. O OpenCV tem
funcionalidades suficientes para exibir uma janela contendo
uma imagem e capturar eventos de teclado e mouse. Isso é suficiente para toda a
interface gráfica.

Tanto o \emph{wrapper} Python do OpenCV quanto o \emph{TensorFlow} representam
dados numéricos, como tensores, usando uma biblioteca numérica para
Python chamada \emph{numpy}. Ela foi usada para várias operações, como
extrair sub imagens
das \emph{frame} de vídeo, após terem sido lidas pelo OpenCV, e fornecer
estes dados para o \emph{TensorFlow}.

Para representar as marcações que o usuário faz nos vídeos, para identificar
onde estão as placas, foi usado o formato \sigla{JSON}{\emph{JavaScript
object notation}, notação de objetos do \emph{JavaScript}}.
Existem boas
bibliotecas em várias linguagens, inclusive Python e Java, o que garante que os
dados poderão ser lidos facilmente.

O Linux foi o único sistema operacional usado, por ser \emph{open source},
gratuitamente acessível e poder ser usado em produtos comerciais, caso surja a
demanda. A distribuição escolhida foi Arch Linux, apesar de não ser
oficialmente suportado pelo projeto \emph{TensorFlow}, por possuir a filosofia
empregar sempre a versão mais atual de todos os componentes, como kernel. O
suporte ao \emph{TensorFlow} é oferecido pela comunidade
\sigla{AUR}{\emph{Arch User
Repository}} (Arch User Repository), que permite a instalação a partir do
código-fonte do \emph{TensorFlow} mesmo antes de uma versão oficial ser liberada.

O sistema de controle de versões usado foi o GIT.
Esse \sigla{VCS}{\emph{Version control system}, sistema de controle de versão}
já é usado no projeto \emph{TensorFlow}.

Para edição de código foi usado exclusivamente vim. Esse editor é sofisticado
e produtivo quando usado com uma linguagem como Python.

\section{Recursos de Hardware}

O único recurso de hardware que foi necessário para este projeto foi
um computador, sendo que dois foram usados:

\begin{itemize}
\item Um notebook Core i7-3632QM com 4 cores (2 threads por core), 16 GiB
	de RAM;
\item Um notebook Core i5-4210U com 2 cores (2 threads por core), 8 GiB de RAM,
	placa de video NVidia GT-750M com 2 GiB de RAM compatível com
	\emph{TensorFlow}.
\end{itemize}

O desenvolvimento do projeto foi quase todo feito usando o Core i7. Assim que
o suporte a GPU foi incluída no código o notebook Core i5 começou a ser usado
concomitantemente.

A \sigla{CPU}{\emph{Central processing unit}, central unificada de
processamento} do computador Core i7, quando usado para treinamento, rapidamente
atinge \SI{85}{\degree} e o desempenho do treinamento reduz consideravelmente,
como consequência da CPU estar atuando para limitar a potência dissipada. Após
o computador rodar por alguns meses desta maneira o computador deixou de ser
confiável, travando com frequência e muitas vezes se recusando a ligar com
sintomas aleatórios. No caso de treinamento prolongado, como foi feito neste
projeto, pode ser necessário conferir a temperatura na qual os componentes
estão rodando, para evitar danos.

\section{Implementação dos Módulos de Software}

Nesta seção está detalhada a implementação de cada módulo de software.

\subsection{Marcador}

\begin{figure}[!htb]
	\centering
	\includegraphics{cap5_tela_marcador.png}
	\caption{Interface com usuário do \emph{Marcador}}
	\label{fig:cap5_tela_marcador}
	A tela está mostrando 3 veículos marcados (próprio autor).
\end{figure}

O software marcador foi construído como um script Python que recebe o nome de
um arquivo de vídeo como parâmetro. Ele permite:

\begin{enumerate}
\item Reproduzir o vídeo;
\item Pausar a reprodução do vídeo
\item Avançar / voltar frame-a-frame;
\item Saltar para uma \emph{frame} digitando-se o número dela;
\item Quando o vídeo está em pausa, permite clicar no vídeo para adicionar um
	marcador de placa;
\item Usar o teclado para mover, rotacionar e alterar o marcador, de forma que ele
	circule corretamente a placa sendo marcada;
\item Digitar o número da placa veicular
\end{enumerate}

Como pode ser visto na figura \ref{fig:cap5_tela_marcador}, o software
marcador permite marcar placas de
carros e motos, e é possível identificar o número da placa. Isto tem como
objetivo permitir que as mesmas marcações possam ser usadas futuramente para
fazer OCR das placas. As marcações são salvas em um arquivo JSON.

Quando uma \emph{frame} possui pelo menos uma placa veicular marcada, todo
o resto da
imagem vai ser considerado pelo software de treinamento como região sem placa.
Por isso, se uma placa for marcada em uma \emph{frame}, devem-se marcar todas
as placas.

\begin{table}
	\center
	\caption{Arquivos de vídeo usados durante o desenvolvimento}
	\renewcommand{\arraystretch}{1.6}
	\begin{tabular}{ccccc}
		\Xhline{6\arrayrulewidth}
		\textbf{Vídeo} &
			\textbf{Resolução} &
			\textbf{FPS} &
			\textbf{Tamanho} &
			\textbf{Duração} \\
		\Xhline{2\arrayrulewidth}
		video1.avi & $480 \times 768$   & 25,0 & 141 MiB & 8:27  \\
		video2.avi & $1080 \times 1920$ & 25,0 & 1,0 GiB & 36:16 \\
		video3.avi & $1080 \times 1920$ & 25,0 & 465 MiB & 8:02  \\
		\Xhline{6\arrayrulewidth}
	\end{tabular}
	\label{tbl:videos}
\end{table}

Todo o desenvolvimento foi feito usando três vídeos. As características dos
vídeos estão listadas na tabela \ref{tbl:videos}, e uma \emph{frame} de
cada vídeo está mostrada na figura \ref{fig:cap5_3_videos}.

\begin{figure}[!htb]
	\centering
	\includegraphics{cap5_3_videos.png}
	\caption{Uma \emph{frame} de cada vídeo usado durante o desenvolvimento}
	\label{fig:cap5_3_videos}
	(próprio autor).
\end{figure}

A tabela \ref{tbl:marc_videos} mostra a quantidade de marcações feitas em cada
vídeo. A maior parte das marcações foram feitas no \emph{video1}. Observar que
existe dois sets de marcações neste vídeo. Uma delas, denominada
\emph{testes}, não foi usada para treinamento, pois está reservada para
testar o desempenho do modelo. Os dados de treinamento estão todos antes do
\emph{frame} número 8000, e todos os dados de teste vêm depois.

\begin{table}
	\center
	\caption{Marcações feitas em cada vídeo}
	\renewcommand{\arraystretch}{1.6}
	\begin{tabular}{c c c}
		\Xhline{6\arrayrulewidth}
		\textbf{Vídeo} &
			\textbf{Quadros Marcados} &
			\textbf{Placas Marcadas} \\
		\Xhline{2\arrayrulewidth}
		video1.avi & 191 & 233 \\
		video1.avi (testes) & 90 & 92 \\
		video2.avi & 71  & 96  \\
		video3.avi & 0   & 0   \\
		\Xhline{6\arrayrulewidth}
		TOTAL      & 262 & 304 \\
	\end{tabular}
	\label{tbl:marc_videos}
\end{table}

\subsection{Gerador de Dados de Treinamento}

\begin{figure}[!htb]
	\centering
	\includegraphics{cap5_tela_gerador.png}
	\caption{Interface gráfica do gerador de dados}
	\label{fig:cap5_tela_gerador}
	A imagem da esquerda mostra o quadro que está sendo processado e mostra as
	regiões onde os exemplos negativos estão sendo coletados. A imagem da
	direita mostra os próprios exemplos negativos que estão sendo enviados para
	o arquivo (próprio autor).
\end{figure}

O software gerador de dados de treinamento também recebe como parâmetro o nome
do arquivo de vídeo onde vai operar, e vai abrir este arquivo de vídeo e as
marcações feitas nele. Este software roda de modo não-interativo,
terminando sem intervenção do usuário quando a geração do conjunto de
treinamento está concluída. A interface gráfica está ilustrada na figura
\ref{fig:cap5_tela_gerador}

O conjunto de treinamento é armazenado em formato binário contendo registros de
tamanho fixo. Cada registro contém a imagem seguida de um rótulo. A
imagem é codificada no formato HWC usando 1 byte por canal de cor,
sendo a cor na ordem RGB. O rótulo representa o valor que a rede neural
deve aprender (número ponto-flutuange de 0 à 1), mas é codificado como um
inteiro de
8 bits sem sinal.  Como a imagem é $40 \times 120 \times 3$ e o rótulo
tem 1 byte, então cada registro possui 14401 bytes.

Este software lê a lista de \emph{frames} marcadas em uma ordem
aleatória. Então vai coletar exemplos de placas veiculares centradas e fora
de centro e exemplos de imagens que não contém placas, e vai salvá-las no
arquivo de saída. A cada um dos exemplos de treinamento este software
também salva um número que é o valor da função que a rede neural deve
produzir para esta placa, conforme equação \ref{eq:funcao_a_modelar}.

Tomando uma placa nas coordenadas $y \times x$, o software primeiro gera um
recorte centrado nessas coordenadas e com o tamanho $40 \times 120$. Este
tamanho é definido em um arquivo de configuração. Todos os exemplos
coletados desta forma recebem o rótulo 1 (codificado como 255).

Para obter os exemplos de placas não centralizadas são coletados recortes
com dimensões $40 \times 120$ de
aproximadamente 4 em 4 \emph{pixels} dentro de uma região próxima da placa. O
algoritmo avança precisamente na taxa de 4 \emph{pixels}, porém adiciona um
número aleatório de -2 à 2 em cada um deles. Isso tem como objetivo evitar de
fornecer dados muito regulares para a rede neural para evitar que ela
aprenda algum truque relacionado à este padrão de distância. A região onde
coleta-se estes exemplos é tal que o centro fica dentro de uma região
$40 \times 120$ centrada na placa do veículo, como ilustrado na figura
\ref{fig:cap5_regiao_coleta_amostras}.

\begin{figure}[!htb]
	\centering
	\includegraphics{cap5_regiao_coleta_amostras.png}
	\caption{Região onde amostras são coletadas}
	\label{fig:cap5_regiao_coleta_amostras}
	Esta região delimita os centros das partições, então pixels fora dessa
	região são coletados. Partições centrados dentro da região delimitada são
	coletados de 4 em 4 pixels, sendo que cada uma dessas regiões é deslocada 2
	pixels para a direita ou para a esquerda. Em verde estão marcados três
	pontos, a, b e c, que serão atribuídos respectivamente aos rótulos 0,
	$^1/_2$ e 1 (próprio autor).
\end{figure}

A região delimitada é exatamente a região na qual o valor é não-zero. O ponto
(c) na figura \ref{fig:cap5_regiao_coleta_amostras} mostra o limiar que está
a $^1/_4$ de distância do centro na direção da largura. Neste ponto a função
começa a decair de 1 para 0 (que são codificados como 255 e 0). O ponto (a)
é onde a função atinge o valor 0, e o ponto (b) está precisamente no
meio do caminho, possuindo o valor 0,5 (codificado como 128).

Após coletar os exemplos de placas contidas na região acima indicada ocorre a
coleta de imagens fora da região, denominada coleta de “exemplos negativos”. A
todos os exemplos coletados dessa maneira é atribuído o rótulo 0.

O algorítmo que coleta esses exemplos passou por vários aprimoramentos. A
versão final, que vai ser descrita, melhorou consideravelmente o desempenho do
treinamento comparado com as versões iniciais.

A primeira, e mais importante otimização foi a eliminação de exemplos
parecidos. Quando são coletados exemplos de regiões da imagem que não são
placas de carro pode-se acabar obtendo um recorte que inclui apenas o asfalto.
Se outro recorte for feito em outra \emph{frame} na mesma região pode acabar
sendo um exemplo muito parecido, como ilustrado na figura
\ref{fig:cap5_coleta_negativa_igual}.

\begin{figure}[!htb]
	\centering
	\includegraphics{cap5_coleta_negativa_igual.png}
	\caption{Exemplo de duas regiões iguais em frames diferentes}
	\label{fig:cap5_coleta_negativa_igual}
	As regiões possuem as mesmas coordenadas, e contém pixels de fundo da
	imagem (próprio autor).
\end{figure}

Por este motivo o software mantém as últimas 100 \emph{frames} processadas.
Quando um “exemplo negativo” vai ser coletado em uma região a imagem é
comparada com os
pixels da mesma região nestas frames que foram armazenadas. O critério de
comparação é:

\begin{equation}
	d_C=\frac{1}{N} \sum_{n=1}^N \sqrt{\left| R_C [n] -I_C [n] \right|}
\end{equation}

Onde $R_C[n]$ é o valor do canal de cor “C” do n-ésimo píxel da partição de
referência, e $I_C[n]$ é o mesmo canal do n-ésimo pixel da
partição que está sendo testada. Quando o valor da média dessa métrica
para os três canais é menor que $\sqrt{8}$ as imagens são consideradas
semelhantes, e a amostra é recusada. Foram
testadas métricas, como média do módulo da diferença, e média da diferença
quadrática, mas tiveram desempenho inferior ao serem comparados usando a
percepção humana como referência. A figura
\ref{fig:cap5_tela_gerador} mostra o efeito desta otimização.
Cada retângulo azul mostra a região onde um exemplo negativo foi coletado.
Observa-se que o asfalto não está sendo coletado.

A segunda otimização foi a distância entre duas amostras negativas. Duas
amostras muito próximas tem pouco valor para treinamento da rede neural devido
à invariância a deslocamento. Por isso, para uma amostra ser aceita como
“exemplo negativo” ela precisa estar a mais de 4 pixels de todos os outros
exemplos coletados.

A terceira otimização foi o processo de escolha das coordenadas onde os
exemplos são coletados. Inicialmente usava-se um grid regular, mas essa
abordagem acabava coletando uma quantidade muito grande de exemplos negativos.

Como isso a rede neural acabava aprendendo a favorecer valores negativos, pois
errar “para mais” acabava sendo punido mais severamente que errar “para menos”
durante a otimização. Para poder balancear a quantidade de exemplos negativos e
positivos foi necessário substituir o grid linear, já que ele favorecia coletar
exemplos negativos na parte superior da imagem, pois a coleta terminava antes
da imagem chegar na parte inferior.

Para resolver isso foi adotado um processo que escolhe coordenadas usando uma
distribuição uniforme e testa se nas coordenadas existe um exemplo negativo
válido. Infelizmente este algoritmo é $O(\infty)$ no pior caso. Por isso um
novo algoritmo de busca
que tem probabilidade próxima de uniforme para a busca, vai sempre encontrar
todas as soluções, e sempre termina precisou ser criado. Este algoritmo é uma
modificação do BFS (\emph{breadth-first search}), no no qual a ordem na
qual a busca é adicionada à pilha é aleatória. O algoritmo pega
de uma pilha uma região
retangular na qual deve encontrar pontos válidos e usa um gerador aleatório
uniforme para escolher coordenadas e testar. Se o teste falhar divide a região
em duas sub-regiões na direção onde a imagem for maior (altura ou largura) e
adiciona as regiões na pilha em ordem aleatória. Cada vez que coordenadas
válidas são encontradas elas são adicionadas na lista de resposta. Se a lista
possui o número solicitado de respostas o algoritmo termina, caso contrário
continua.

\begin{table}
	\center
	\caption{Exemplos de treinamento gerados para cada vídeo}
	\renewcommand{\arraystretch}{1.6}
	\begin{tabular}{c c c p{2.5cm} p{2.5cm}}
		\Xhline{6\arrayrulewidth}
		\textbf{Entrada} &
			\textbf{Saída} &
			\textbf{Tamanho} &
			\textbf{Registros \newline Gerados} \\
		\Xhline{2\arrayrulewidth}
		video1.avi & video1.nn1.bin & 1,1 GiB  & 64.714  \\
		video2.avi & video2.nn1.bin & 901 MiB  & 89.672  \\
		\Xhline{6\arrayrulewidth}
		TOTAL      &                & 2,07 GiB & 154.386 \\
	\end{tabular}
	\label{tbl:marc_videos}
\end{table}

Um script escrito em bash foi escrito partir um arquivo destes em arquivos
menores, contendo 128 amostras cada. Isso gerou 630 arquivos a partir de
\emph{video1.nn1.bin} e 240 arquivos a partir de \emph{video2.nn1.bin}.

Finalmente, um diretório foi criado contendo links simbólicos para todos os 869
arquivos. Este diretório é o resultado final da geração de exemplos, e contém
todas as amostras coletadas de todos os vídeos.

\subsection{Treinador}

\begin{figure}[!htb]
	\centering
	\includegraphics{cap5_tela_treinador.png}
	\caption{Página do \emph{TensorBoard} mostrando progresso do treinamento}
	\label{fig:cap5_tela_treinador}
	Esta é uma página \emph{web} que permite observar a evolução do
	treinamento (próprio autor).
\end{figure}

O treinador é um módulo de software onde se configura uma topologia de rede
neural convolucional, que é treinada usando os dados produzidos pelo gerador de
dados de treinamento.

Para realizar o treinamento tanto da rede neural um grafo apresentado na figura
\ref{fig:cap5_grafo_treinamento} foi construído.

\begin{figure}[!htb]
	\centering
	\includegraphics{cap5_grafo_treinamento.png}
	\caption{Grafo usado para treinamento da rede neural convolucional}
	\label{fig:cap5_grafo_treinamento}
	Esta imagem mostra o grafo do \emph{TensorFlow} usado para treinar
	a rede neural, que é o bloco \emph{NN1} (próprio autor).
\end{figure}

O bloco “FileRead” possui uma lista com todos os arquivos de treinamento
gerados pelo gerador de exemplos de treinamento. Essa lista é embaralhada, e
então cada um dos registros é lido em sequência. A imagem é representada por um
tensor de inteiros de 8 bits com dimensões $40 \times 120 \times 3$, e o
rótulo da imagem é representado com um tensor unidimensional com dimensão $1$.

Como a rede neural está sendo treinada usando apenas dois vídeos existe o risco
da rede neural treinada fique muito sensível a fatores como a iluminação. Para
impedir que isso aconteça as imagens são distorcidas durante o treinamento no
bloco de distorções, que está ilustrado na figura \ref{fig:cap5_distorcao}.
A imagem é convertida para ponto-flutuante com canais no intervalo $[0;1]$,
conforme requerido pelos primitivos do próprio \emph{TensorFlow}.
No final das distorções a imagem é convertida
novamente para inteiro de 8 bits por canal. O motivo para isso é fazer com o
que o bloco “NN1”, que implementa a rede neural seja o mais otimizado possível
para classificar imagens, não tanto para treinamento.

\begin{figure}[!htb]
	\centering
	\includegraphics{cap5_distorcao.png}
	\caption{\emph{Pipeline} de distorção de imagens}
	\label{fig:cap5_distorcao}
	A imagem é convertida para ponto-flutuante com cada canal no intervalo
	$[0;1]$, é adicionado ruído normal, a saturação, brilho e contraste são
	alterados conforme uma distribuição normal. O número é então convertido
	novamente para inteiro de 8 bits por canal (próprio autor).
\end{figure}

O bloco “Noise” adiciona ruído normal à imagem somando-a com um tensor com as
mesmas dimensões onde o valor de cada canal é gerado por uma distribuição
normal de média 0 e desvio padrão 0,06.

O bloco “Saturation” altera a saturação entre 0,2 a 1,5, sendo que 1 representa
manter a saturação inalterada. O bloco “Brightness” faz uma alteração no brilho
aumentando ou diminuindo em até 40\%. O bloco “Contrast” faz uma alteração de
contraste entre 0,6 e 1,4, sendo que 1 representa manter o brilho. As escolha
do valor em todos os casos é aleatória de acordo com distribuição uniforme.

O resultado das distorções, juntamente com o rótulo da imagem são fornecidos
para o bloco “shuffle\_batch”. Este bloco tem a função de embaralhar os
dados e agrupar imagens em batches. Este bloco usa 8 threads para consumir
os dados dos blocos anteriores, o que significa que leitura e shuffling vai
ocorrer em paralelo. Quando 2560 imagens são lidas elas são embaralhadas
elas são agrupadas em batches de 256 imagens, gerando tensores com
dimensões $256 \times 40 \times 120 \times 3$. Como cada arquivo contém
128 registros, então as imagens
sendo lidas contém, no pior caso, uma amostra aleatória proveniente de 20
arquivos. Os arquivos em sí são lidos em ordem aleatória, o que gera uma
amostragem razoavelmente diversa das informações de entrada.

Quanto maior o tamanho do batch maior a precisão da estimativa de erro,
portanto este número é o maior possível suportado pelo hardware onde o
treinamento está sendo realizado.

Se o treinamento for rodado por tempo suficiente, todas as imagens serão
lidas. Se o treinador precisar de mais exemplos as mesmas imagens serão lidas
novamente, em ordem diferente. Porém, graças ao sistema de distorção de
imagens que é parte do pipeline de leitura, uma mesma imagem lida múltiplas
vezes muito provavelmente não chegará idêntica na rede neural.

Os tensores que saem do shuffler são fornecidos a rede neural, que está no
bloco “NN1”, que está ilustrado na figura \ref{fig:cap5_cnn}.

\begin{figure}[!htb]
	\centering
	\includegraphics{cap5_cnn.png}
	\caption{Pipeline descrevendo a rede neural convolucionao profunda}
	\label{fig:cap5_cnn}
	(próprio autor)
\end{figure}

Pode-se ver pela espessura das linhas que conectam os blocos que a quantidade
de informação reduz ligeiramente em cada etapa. Como a rede neural precisa
de um tempo de propagação baixo algumas técnicas foram usadas para
reduzir o número de computações.

A primeira estratégia é a redução agressiva do tamanho dos tensores nas duas
primeiras camadas, inspirado em \cite{szegedy2015going}. \emph{Stride} é usado
para fazer um resampling da imagem. Este método usa mais CPU do que resampling
por métodos como bilinear, porém permite acesso à textura da imagem original. A
primeira camada usa um kernel de $2 \times 2$ ao invéz de $3 \times 3$,
reduzindo o número de multiplicações de 9 para 4.

Não está sendo usada nenhuma convolução maior que $3 \times 3$.
\cite{szegedy2015going} defende que, pelo menos em alguns casos,
convoluções $5 \times 5$ podem ser substituídas por duas convolução
sucessivas $3 \times 3$ enquanto reduz o número de
multiplicações de $25$ para $2 \cdot 9=18$.

\begin{table}
	\center
	\caption{Camadas da Rede Neural Implementada}
	\renewcommand{\arraystretch}{1.6}
	\begin{tabular}{c c c c}
		\Xhline{6\arrayrulewidth}
		\textbf{Nome} &
			\textbf{Tensor de Entrada} &
			\textbf{Operação} &
			\textbf{DOF} \\
		\Xhline{2\arrayrulewidth}
		M1 & $256 \times 40 \times 120 \times 3$ & 6 cv2x2 s2x2 relu  & 78   \\
		M2 & $256 \times 20 \times 60 \times 6$  & 8 cv3x3 s2x2 relu  & 440  \\
		M3 & $256 \times 10 \times 30 \times 8$  & 8 cv3x3 relu       & 584  \\
		M4 & $256 \times 10 \times 30 \times 8$  & 8 cv3x3 relu       & 584  \\
		M5 & $256 \times 10 \times 30 \times 8$  & 8 cv3x3 relu mp2x2 & 584  \\
		M6 & $256 \times 5  \times 15 \times 8$  & 6 cv3x3 relu       & 438  \\
		M7 & $256 \times 5  \times 15 \times 6$  & 3 cv3x3 relu mp2x2 & 165  \\
		-  & $256 \times 3  \times 8 \times 3$   & flatten            & 0    \\
		FC32 & $256 \times 72$                   & fc32 relu          & 2.336\\
		FC1 & $256 \times 32$                    & fc1 linear         & 33   \\
		SAÍDA & $256$                            &                    &      \\
		\Xhline{6\arrayrulewidth}
		TOTAL & & & 5.242 \\
	\end{tabular}
	\label{tbl:marc_videos}
\end{table}

Nas últimas camadas está sendo usado maxpool $2 \times 2$. Esta estratégia é
computacionalmente mais cara que usar convolução com stride maior que 1, pois
de cada 4 valores 3 deles são descartados. Como as imagens são menores no final
da rede neural optou-se pela operação mais cara. O efeito em tempo de
propagação é negligenciável, o efeito na qualidade da detecção não foi medido.

Outro ponto importante é o controle do número de graus de liberdade. Os
primeiros modelos tinham mais de 50.000 graus de liberdade, e este valor foi
reduzido para 5.242. Menos parâmetros implica várias vantagens, como
treinamento mais rápido, menos uso de memória, capacidade de usar
\emph{batches} maiores e menor chance de \emph{overfitting}.

O processo que mais ajudou a aprimorar sucessivamente os hiperparâmetros da
rede neural foi procurar uma rede neural com menos parâmetros que tenha
desempenho comparável. Quando esta estratégia foi adotada a produtividade no
que diz respeito a otimização destes parâmetros teve um salto considerável.

A saída da rede neural, denominado “A”, é um tensor contendo um valor para cada
uma das imagens, portanto um com dimensão 256. Para o treinamento a saída da
rede neural vai ser comparado com os rótulos que estavam nos dados de
treinamento, representado pelo tensor “B”, e também forma um tensor
monodimensional com dimensão 256. O erro entre as duas medidas é mapeado para
um único escalar, denominado perda, ou loss, pela metade da normal L2 entre os
dois tensores sem a raíz quadrada:

\begin{equation}
	loss=\frac{1}{2} \sum_{n=1}^N \left( A[n] - B[n] \right)^2
\end{equation}

Esta medida foi usado porque é calculada eficiente pelo \emph{TensorFlow}
através da função \textbf{tf.nn.l2\_loss(t, name=None)}. O software de
treinamento usa um otimizador
Adam \cite{kingma2014adam} para atualizar os parâmetros treináveis da rede
neural de forma a minimizar este valor.

As figuras \ref{fig:cap5_train_histogram} e \ref{fig:cap5_erro_treinamento}
mostram a evolução do treinamento durante uma sessão de 25
horas treinadas em um notebook Core i5 usando \sigla{GPU}{\emph{Graphical
processing unit}, unidade gráfica de processamento}. Durante este tempo foram
consumidos mais de 66 milhões de imagens, o que consumiu todos os 2 GiB de
dados e seus 154.386 exemplos de imagens cerca de 428 vezes.

\begin{figure}[!htb]
	\centering
	\includegraphics{cap5_train_histogram.png}
	\caption{Histograma temporal do erro classificação}
	\label{fig:cap5_train_histogram}
	Pode-se observar os erros se concentrando cada vez mais próximo de 0 a
	medida que o treinamento avança. A escala indicada são batches de 256
	imagens, portanto este gráfico o resultado após mais de 66 milhões de
	imagens (próprio autor).
\end{figure}

\begin{figure}[!htb]
	\centering
	\includegraphics{cap5_erro_treinamento.png}
	\caption{Gráfico do erro de treinamento}
	\label{fig:cap5_erro_treinamento}
	Evolução da função de perda que o otimizador está minimizando, amortecido
	usando método média local. A abscissa é o número do batch, e cada batch
	possui 256 imagens (próprio autor).
\end{figure}

\subsection{Player}

\begin{figure}[!htb]
	\centering
	\includegraphics{cap5_tela_player.png}
	\caption{Interface com o usuário do módulo \emph{Player}}
	\label{fig:cap5_tela_player}
	No momento que a tela foi capturada o software está exibindo um vídeo
	enquanto destaca as placas que está localizando (próprio autor).
\end{figure}

O quarto módulo de software usa a rede neural treinada para reproduzir o
conteúdo de um vídeo ou da câmera enquanto destaca as placas de carros
localizadas. A figura \ref{fig:cap5_tela_player} mostra a interface do usuário
no momento em que está exibindo um vídeo quando três veículos estão passando.
As placas dos mesmos estão sendo destacadas.

O software implementa o método de classificação proposto nesta monografia por
completo. O método de binarização é um limiar simples em 0,75. Este valor
permite que quando a placa está transitando de uma partição para outra, ambas
as placas serão iluminadas momentaneamente antes da placa que ficou para trás
apagar.

Não foi implementado nenhum método para usar a correlação temporal das
informações coletadas.

Para o \emph{video1.avi} a rede neural é executada 288 vezes para frame do
vídeo, sendo que uma linha completa contendo 12 partições é fornecido para a
rede neural por vez.

O software é capaz de operar em qualquer vídeo que possa ser lido pelo
\emph{OpenCV3} e pode usar a imagem da câmera do computador.

\section{Experimentos e Desempenho}

Nesta sessão serão apresentados os experimentos realizados com a implementação
do software e os seus resultados.

\subsection{Desempenho de Localização}

Como o método de localização proposto é aproximado, a abordagem para teste de
desempenho é baseada na contagem de placas corretamente ou
incorretamente localizadas. Para que uma placa seja considerada corretamente
localizada basta que ela esteja a menos de 60 pixels de distância na largura e
menos de 20 pixels na altura. Estes limites foram escolhidos por serem metade
do tamanho das partições, que representa a distância na qual a função que está
sendo modelada pela rede neural deve ir à zero.
	
Para a medição de desempenho serão usadas as métricas \emph{precision}
(equação \ref{eq:precision}) e \emph{recall} (equação \ref{eq:recall}).

\noindent\begin{minipage}{.5\linewidth}
	\begin{equation} \label{eq:precision}
		\text{Precision} = \frac{vp}{vp + fp}
	\end{equation}
\end{minipage}
\begin{minipage}{.5\linewidth}
	\begin{equation} \label{eq:recall}
		\text{Recall} = \frac{vp}{vp + fn}
	\end{equation}
\end{minipage}

Onde $vp$ é o número de verdadeiros positivos, $fp$ é o número de falsos
positivos e $fn$ é o número de falsos negativos. Estes dados serão obtidos da
seguinte maneira:

\begin{itemize}

\item Quando o software identificar uma placa em um ponto do \emph{frame} e
	houver uma placa entre as manualmente marcadas, tal que a distância entre
	os centros  das duas seja inferior ao erro, será contabilizado um
	\emph{verdadeiro positivo}, ou \emph{tp} (figura \ref{fig:cap5_tp-fn-fp},
	placa ``a'').
	A placa é removida da lista de placas manualmente marcadas para não
	ser usada novamente, pois já foi localizada (figura
	\ref{fig:cap5_tp-fn-fp}, placa ``b'').;

\item Se no caso anterior não houver placa manualmente marcada tal que o erro
	seja inferior ao máximo será contabilizado um falso positivo, ou \emph{fp}
	(figura \ref{fig:cap5_tp-fn-fp}, placa ``c'');

\item as placas que foram manualmente marcadas e não forem localizadas são
	contabilizadas como falso negativos, ou \emph{fn} (figura
	\ref{fig:cap5_tp-fn-fp}, placa ``A'').

\end{itemize}

\begin{figure}[!htb]
	\centering
	\includegraphics[scale=0.7]{cap5_tp-fn-fp.png}
	\caption{Contagem de acertos e erros de localização}
	\label{fig:cap5_tp-fn-fp}
	Cada círculo representa o centro de uma placa. Constam 1 \emph{verdadeiro
	positivo} (B/a), 2 falsos positivos (b, c), e um \emph{falso negativo} (A).
	Próximo a placa ``B'' foram localizadas duas placas, mas
	só há uma, então uma delas é contada como \emph{falso positivo}
	(próprio autor).
\end{figure}

Para usar usar corretamente o método proposto é preciso escolher um limiar de
classificação. Se o escore de uma partição for maior que este limiar
considera-se que a partição possui uma placa veicular. Como este escore afeta
as métricas \emph{precision} e \emph{recall}, foi feito um levantamento da
\emph{curva precision-recall} para determinar o valor mais apropriado de limiar
a ser usado. A tabela \ref{tbl:curva-pr} e o gráfico na figura
\ref{fig:cap5_res_curva-pr} esta curva para os dados de teste. Apenas para
referência, o gráfico também foi calculado para o set de treinamentos, por ser
maior, gerando o gráfico da figura \ref{fig:cap5_res_curva-pr-train}.

\noindent\begin{minipage}{.5\linewidth}
	\begin{equation} \label{eq:precision}
		\text{Precision} = \frac{vp}{vp + fp}
	\end{equation}
\end{minipage}
\begin{minipage}{.5\linewidth}
	\begin{equation} \label{eq:recall}
		\text{Recall} = \frac{vp}{vp + fn}
	\end{equation}
\end{minipage}

\begin{table}
	\center
	\caption{Erros e acertos de localização}
	\renewcommand{\arraystretch}{1.6}
	\begin{tabular}{ccc}
		\Xhline{6\arrayrulewidth}
		\textbf{Limiar} &
			\textbf{Recall} &
			\textbf{Precision} \\
		\Xhline{2\arrayrulewidth}
			0    & 100,0\%  &  3,44\% \\
			0,1  & 100,0\%  &  53,8\% \\
			0,14 & 100,0\%  &  57,9\% \\
			0,2  & 100,0\%  &  67,2\% \\
			0,3  & 100,0\%  &  78,0\% \\
			0,5  & 100,0\%  &  94.8\% \\
			0,7  & 100,0\%  &  94,8\% \\
			\textbf{0,8}  &  \textbf{96,7\%}  &  \textbf{98,9\%} \\
			0,9  &  94,6\%  &  98,9\% \\
			0,93 &  91,3\%  &  98,8\% \\
			0,97 &  80,4\%  &  98,7\% \\
			0,99 &  65,2\%  &  98,4\% \\
			1,01 &  19,6\%  & 100,0\% \\
		\Xhline{6\arrayrulewidth}
	\end{tabular}
	\label{tbl:curva-pr}
\end{table}

\begin{figure}[!htb]
	\centering
	\includegraphics[scale=1]{cap5_res_curva-pr-train.png}
	\caption{Gráfico da curva Precision-Recall para amostras de treinamento}
	\label{fig:cap5_res_curva-pr-train}
	Oserva-se uma curva semelhante a curva PR para o set de testes. Como este
	set é maior a forma é mais suave (próprio autor).
\end{figure}

\begin{figure}[!htb]
	\centering
	\includegraphics[scale=1]{cap5_res_curva-pr.png}
	\caption{Gráfico da curva Precision-Recall}
	\label{fig:cap5_res_curva-pr}
	Pode-se observar como o limiar de classificação afeta \emph{precision} e
	\emph{recall} (próprio autor).
\end{figure}

Tendo sido escolhido o limiar 0.8 para aceitar uma placa, as métricas
\emph{precision} e \emph{recall} foram calculados para todos os conjuntos
de dados marcados disponíveis, tanto de teste quanto de treinamento.
Os resultados obtidos estão apresentados na tabela
\ref{tbl:vp-fp-fn}. Os indicadores de desempenho foram melhores no
conjunto de dados de teste do que no conjunto de treinamento. Isso
se deve ao fato de que a implementação, como foi feita, não detecta
placas que estejam próximos a borda direita e
inferior da imagem. O set de testes foi preparado tomando cuidado para não usar
frames onde veículos se encontrem nesta região.

O resultado do \emph{precision}
no video2 foi apenas 59,0\%, causado pela quantidade excessiva de falso
positivos. Observou-se que uma única das sessões da imagem que pertence ao
asfalto, foi marcada em aproximadamente 50\% das frames como placa, gerando
falso positivos. Atribui-se este problema a resolução deste vídeo, que é $1090
\times 1920$, e portanto requer uma grande quantidade de marcações para que
seja coletada uma quantidade suficiente de amostras de ``não-placa'' para
eliminar este problema, já que o software de coleta de exemplos de treinamento
tenta balancear a quantidade de exemplos positivos e negativos.


\begin{table}
	\center
	\caption{Erros e acertos de localização}
	\renewcommand{\arraystretch}{1.6}
	\begin{tabular}{p{4.5cm} c c c c c}
		\Xhline{6\arrayrulewidth}
		\textbf{Vídeo} &
			\textbf{VP} &
			\textbf{FP} &
			\textbf{FN} &
			\textbf{Precision} &
			\textbf{Recall} \\
		\Xhline{2\arrayrulewidth}
		\textbf{video1.avi (teste)} &
			\textbf{89} &
			\textbf{1} &
			\textbf{3} &
			\textbf{98,9\%} &
			\textbf{96,7\%} \\
		video1.avi (treinamento) & 217 & 11 & 16 & 95,2\% & 93,1\% \\
		video2.avi (treinamento) &  92 & 64 &  4 & 59,0\% & 95,8\% \\
		\Xhline{6\arrayrulewidth}
	\end{tabular}
	\label{tbl:vp-fp-fn}
\end{table}

\subsection{Função}

Como o método proposto envolve fazer a regressão de uma função, as primeiras
medições aqui apresentadas estão mostrando o resultado desse modelamento. As
figuras \ref{fig:cap5_res_func_1} e \ref{fig:cap5_res_func_2} mostram 3 carros,
uma moto e um ônibus, sendo que o classificador foi aplicado pixel-a-pixel na
região próxima da placa, e o seu valor representado em um gráfico 3D. Este
gráfico pode ser comparado com o gráfico mostrado na figura
\ref{fig:func_a_modelar_3d}, que mostra o gráfico ideal.

Estes dados foram todos levantados usando o vídeo3.avi, que não usado para
treinamento. As cinco frames foram escolhidos ao acaso, garantindo-se apenas
que as cinco imagens sejam suficientemente diferentes umas das outras.

A capacidade de produzir estes gráficos foi incluído no software
``\emph{player}'', e pode ser invocada para qualquer frame de qualquer vídeo.

\begin{figure}[!htb]
	\centering
	\includegraphics{cap5_res_func_1.png}
	\caption{Gráfico 3D da função aplicada a carros}
	\label{fig:cap5_res_func_1}
	São destacados três carros, um com boa textura, placa razoavelmente
	visível, outra com imagem com perda resultante de compactação, e um
	terceiro em posição, e cores diferentes, inclusive usando letras mais
	claras que o frame da placa. O gráfico mostra o resultado do classificador
	quando ele está centrada em cada um dos pixels próximo à placa.
	As abiscissas referem-se as coordenadas do canto superior-esquerdo da
	partição, não ao seu centro (próprio autor).
\end{figure}

\begin{figure}[!htb]
	\centering
	\includegraphics{cap5_res_func_2.png}
	\caption{Gráfico 3D da função aplicada a moto e ônibus}
	\label{fig:cap5_res_func_2}
	Observa-se uma moto e um veículo e o gráfico resultante de aplicar o
	classificador em torno das placas. As abiscissas referem-se as coordenadas
	do canto superior-esquerdo da partição, não ao seu centro (próprio autor).
\end{figure}

A figura \ref{fig:cap5_res_func_frame} mostra a rede neural sendo aplicada
na frame 11.197 do video1. Esta frame não faz parte do conjunto de treinamento.
Pode-se verificar claramente onde estão as duas placas veiculares.

\begin{figure}[!htb]
	\centering
	\includegraphics{cap5_res_func_frame.png}
	\caption{Gráfico 3D mostrando a função aplicada à uma frame inteira}
	\label{fig:cap5_res_func_frame}
	Pode-se ver claramente onde as placas dos dois veículos se encontram.
	(próprio autor).
\end{figure}

\subsection{Tempo de Classificação}
Para os testes de tempo de classificação foi usado o módulo \emph{player}. Este
módulo mostra uma janela com o vídeo sendo classificado enquanto envia para
\emph{stdout} uma série de informações de desempenho. Estes dados foram
usados nas medições relacionadas a tempo de classificação.


\begin{table}
	\center
	\caption{Taxa de Frames por Segundo}
	\renewcommand{\arraystretch}{1.6}
	\begin{tabular}{ccccc}
		\Xhline{6\arrayrulewidth}
		\textbf{Vídeo} &
			\textbf{Resolução} &
			\textbf{FPS (GPU)} &
			\textbf{FPS (CPU)} &
			\textbf{Melhoria} \\
		\Xhline{2\arrayrulewidth}
		camera     & $480  \times 640$  & 15,8 & 9.12 & 73\%  \\
		video1.avi & $480  \times 768$  & 14,8 & 8,08 & 84\%  \\
		video2.avi & $1080 \times 1920$ & 3,29 & 1,51 & 117\% \\
		video3.avi & $1080 \times 1920$ & 3,34 & 1,51 & 121\%  \\
		\Xhline{6\arrayrulewidth}
	\end{tabular}
	\label{tbl:player_fps}
\end{table}

Observa-se na tabela \ref{tbl:player_fps} que o \emph{player} não consegue
atingir a taxa de 25 frames por segundo
de nenhum dos três vídeos, mesmo com uso de GPU. A taxa de frames
aumenta consideravelmente quando a GPU é usada. Entre os casos
testados o ganho é maior quando a resolução do vídeo aumenta. A figura
\ref{fig:cap5_player_fps} mostra o resultado em forma de gráfico.

\begin{figure}[!htb]
	\centering
	\includegraphics{cap5_player_fps.png}
	\caption{Taxa de frames por segundo no módulo \emph{Player}}
	\label{fig:cap5_player_fps}
	Gráfico mostrando desempenho em frames por segundo para localizar placas na
	câmera VGA do computador, e nos três vídeos. Computador é um Core i5 com
	placa de vídeo GTX 750M. Maior é melhor (próprio
	autor).
\end{figure}

O \emph{player} envia para o \emph{stdout} o tempo gasto em cada uma das
tarefas durante o processamento da frame. Coletando-se estes dados e
calculando-se as médias do tempo gasto em cada tarefa foi gerado o
gráfico na figura \ref{fig:cap5_frame_benchmark}.

\begin{figure}[!htb]
	\centering
	\includegraphics{cap5_frame_benchmark.png}
	\caption{Tempo gasto em cada subtarefa no processamento de uma
		\emph{frame}}
	\label{fig:cap5_frame_benchmark}
	Gráfico mostra que, ao processar um vídeo $1080 \times 1920$ a grande
	maioria do tempo é gasto executando o grafo do \emph{TensorFlow}. Outras
	tarefas executadas pelo código em Python consomem um percentual
	pequeno do tempo da frame. Computador é um Core i5 com
	placa de vídeo GTX 750M (próprio autor).
\end{figure}

Pode-se ver claramente que a vasta maioria do tempo é gasta executando os
grafos do \emph{TensorFlow}. Também foi possível confirmar que o ganho do
desempenho no caso do uso do GPU foi consequência do \emph{TensorFlow}
consumir menos tempo para executar. Observou-se que na implementação CPU
o \emph{TensorFlow} tomou 94,9\% do tempo de processamento da frame, e no
caso que usa GPU tomou 88,1\%.
 % implementacao e experimentos
%Exemplo de capitulo

\chapter{Conclusões}


 % conclusões

%--------------------------------------
%Elementos Pós-Textuais
%--------------------------------------

%Bibliografia, gerada automaticamente a partir do arquivo .bib em conjunto com as citações presentes no texto.
\bibliography{bib/bibliografia}

%Apendices. Use caso necessário. Todos capítulos após o comando \apendice serão tratados como Apêndices. 
%\apendice
%\chapter{Titulo do Apêndice}

\center

\textbf{M1: 6 convoluções $2 \times 2$ aplicadas sobre uma entrada que possui
3 canais}

Filtro 1
{ \small
\[
\begin{pmatrix}
  -0.136 & -0.147 \\
  -0.149 & -0.153 \\
\end{pmatrix}
\begin{pmatrix}
  -0.206 & -0.314 \\
  -0.227 & -0.216 \\
\end{pmatrix}
\begin{pmatrix}
  -0.294 & -0.377 \\
  -0.344 & -0.41 \\
\end{pmatrix}
\]
}

Filtro 2
{ \small
\[
\begin{pmatrix}
  -7.47 & -6.94 \\
  -6.23 & -7.03 \\
\end{pmatrix}
\begin{pmatrix}
  -11.2 & -11.4 \\
  -11.2 & -10.6 \\
\end{pmatrix}
\begin{pmatrix}
  -9.88 & -8.68 \\
  -9.93 & -9.67 \\
\end{pmatrix}
\]
}

Filtro 3
{ \small
\[
\begin{pmatrix}
  -0.74 & -0.746 \\
  -0.748 & -0.733 \\
\end{pmatrix}
\begin{pmatrix}
  0.355 & 0.345 \\
  0.331 & 0.345 \\
\end{pmatrix}
\begin{pmatrix}
  0.41 & 0.414 \\
  0.388 & 0.397 \\
\end{pmatrix}
\]
}

Filtro 4
{ \small
\[
\begin{pmatrix}
  -0.297 & -0.0571 \\
  -0.00531 & 0.178 \\
\end{pmatrix}
\begin{pmatrix}
  -0.247 & -0.0378 \\
  0.053 & 0.245 \\
\end{pmatrix}
\begin{pmatrix}
  -0.205 & -0.00137 \\
  0.0569 & 0.215 \\
\end{pmatrix}
\]
}

Filtro 5
{ \small
\[
\begin{pmatrix}
  -0.0539 & -0.108 \\
  -0.0727 & -0.113 \\
\end{pmatrix}
\begin{pmatrix}
  -0.0388 & -0.0852 \\
  -0.0445 & -0.0968 \\
\end{pmatrix}
\begin{pmatrix}
  -0.0323 & -0.0797 \\
  -0.0342 & -0.0778 \\
\end{pmatrix}
\]
}

Filtro 6
{ \small
\[
\begin{pmatrix}
  -0.0133 & 0.233 \\
  -0.205 & 0.107 \\
\end{pmatrix}
\begin{pmatrix}
  -0.103 & 0.215 \\
  -0.286 & 0.0714 \\
\end{pmatrix}
\begin{pmatrix}
  -0.0455 & 0.21 \\
  -0.231 & 0.09 \\
\end{pmatrix}
\]
}
\textbf{M2: 8 convoluções $3 \times 3$ aplicadas sobre uma entrada que 
possui 6 canais}

Filtro 1
{ \small
\[
\begin{pmatrix}
  0.503 & 0.0624 & -0.123 \\
  0.278 & -0.536 & -0.392 \\
  0.0932 & 0.0148 & -0.13 \\
\end{pmatrix}
\begin{pmatrix}
  0.434 & 0.0444 & 0.128 \\
  0.294 & -0.67 & -0.319 \\
  0.123 & -0.212 & -0.0443 \\
\end{pmatrix}
\begin{pmatrix}
  0.00874 & 0.00071 & 0.0132 \\
  -0.0179 & 0.0339 & 0.0291 \\
  0.0325 & 0.0359 & 0.0513 \\
\end{pmatrix}
\]
\[
\begin{pmatrix}
  0.136 & 0.204 & 0.0935 \\
  0.256 & 0.223 & 0.042 \\
  0.139 & 0.00817 & -0.0169 \\
\end{pmatrix}
\begin{pmatrix}
  1.02 & 0.305 & -0.0393 \\
  0.434 & -0.741 & -0.963 \\
  0.227 & -0.0978 & -0.158 \\
\end{pmatrix}
\begin{pmatrix}
  0.121 & 0.173 & -0.0611 \\
  0.0229 & -0.0496 & -0.199 \\
  -0.0141 & 0.065 & 0.0134 \\
\end{pmatrix}
\]
}

Filtro 2
{ \small
\[
\begin{pmatrix}
  0.422 & 0.36 & 0.121 \\
  0.435 & 0.255 & 0.13 \\
  0.354 & 0.148 & 0.0757 \\
\end{pmatrix}
\begin{pmatrix}
  -0.518 & -0.737 & 0.049 \\
  -0.44 & -0.703 & -0.546 \\
  -0.404 & -0.444 & -0.491 \\
\end{pmatrix}
\begin{pmatrix}
  -0.509 & -0.86 & -0.697 \\
  -0.698 & -1.0 & -0.879 \\
  -0.442 & -0.439 & -0.445 \\
\end{pmatrix}
\]
\[
\begin{pmatrix}
  0.0997 & 0.238 & 0.0431 \\
  0.312 & -0.0242 & -0.464 \\
  0.0897 & 0.0731 & 0.102 \\
\end{pmatrix}
\begin{pmatrix}
  0.501 & 0.062 & -0.463 \\
  0.322 & 0.00757 & -0.36 \\
  0.0937 & 0.019 & 0.00237 \\
\end{pmatrix}
\begin{pmatrix}
  0.23 & 0.139 & -0.11 \\
  0.00407 & 0.184 & 0.183 \\
  -0.274 & -0.126 & 0.101 \\
\end{pmatrix}
\]
}

Filtro 3
{ \small
\[
\begin{pmatrix}
  -0.293 & -0.0129 & -0.157 \\
  -0.374 & -0.44 & -0.395 \\
  -0.105 & 0.0913 & 0.186 \\
\end{pmatrix}
\begin{pmatrix}
  -0.0709 & -0.00436 & 0.158 \\
  0.00838 & 0.179 & -0.0749 \\
  0.254 & 0.117 & 0.227 \\
\end{pmatrix}
\begin{pmatrix}
  0.0453 & 0.0494 & -0.228 \\
  -0.0412 & -0.105 & -0.136 \\
  -0.0175 & -0.11 & -0.199 \\
\end{pmatrix}
\]
\[
\begin{pmatrix}
  -0.22 & 0.00589 & 0.102 \\
  -0.213 & -0.296 & -0.0955 \\
  -0.116 & -0.167 & -0.0885 \\
\end{pmatrix}
\begin{pmatrix}
  -0.852 & -0.472 & -0.151 \\
  -0.401 & -0.739 & -0.358 \\
  0.0371 & -0.1 & -0.309 \\
\end{pmatrix}
\begin{pmatrix}
  0.0537 & -0.378 & -0.017 \\
  0.222 & -0.187 & -0.132 \\
  0.116 & -0.136 & -0.213 \\
\end{pmatrix}
\]
}

Filtro 4
{ \small
\[
\begin{pmatrix}
  0.603 & 0.447 & 0.095 \\
  0.54 & 0.551 & 0.255 \\
  0.0226 & -0.0676 & 0.0674 \\
\end{pmatrix}
\begin{pmatrix}
  0.236 & 0.284 & 0.141 \\
  0.48 & 0.541 & 0.203 \\
  0.0571 & 0.0207 & 0.0361 \\
\end{pmatrix}
\begin{pmatrix}
  -0.0194 & -0.03 & -0.0282 \\
  -0.0243 & -0.0311 & -0.0239 \\
  -0.0059 & 0.00542 & 0.00164 \\
\end{pmatrix}
\]
\[
\begin{pmatrix}
  0.0226 & 0.0267 & 0.0538 \\
  0.121 & 0.105 & 0.0155 \\
  0.0795 & -0.0231 & -0.0252 \\
\end{pmatrix}
\begin{pmatrix}
  0.682 & 0.484 & 0.213 \\
  0.672 & 0.658 & 0.288 \\
  -0.0506 & -0.0879 & 0.112 \\
\end{pmatrix}
\begin{pmatrix}
  0.0695 & 0.109 & 0.0828 \\
  0.0203 & 0.0873 & 0.0938 \\
  -0.0111 & -0.00605 & 0.035 \\
\end{pmatrix}
\]
}

Filtro 5
{ \small
\[
\begin{pmatrix}
  -0.092 & 0.646 & 0.41 \\
  -0.589 & 0.252 & -0.173 \\
  -0.297 & 0.257 & -0.127 \\
\end{pmatrix}
\begin{pmatrix}
  0.103 & 0.787 & 0.521 \\
  -0.51 & 0.488 & -0.076 \\
  -0.372 & 0.216 & -0.234 \\
\end{pmatrix}
\begin{pmatrix}
  0.00942 & -0.0227 & -0.00383 \\
  0.0429 & -0.0109 & 0.0213 \\
  0.00296 & 0.00327 & 0.0176 \\
\end{pmatrix}
\]
\[
\begin{pmatrix}
  -0.0595 & 0.126 & 0.189 \\
  -0.0245 & 0.142 & 0.124 \\
  -0.181 & 0.0659 & -0.0139 \\
\end{pmatrix}
\begin{pmatrix}
  -0.218 & 1.19 & 0.116 \\
  -0.854 & 0.332 & -0.436 \\
  -0.218 & 0.228 & -0.093 \\
\end{pmatrix}
\begin{pmatrix}
  -0.14 & 0.0505 & 0.17 \\
  -0.236 & -0.124 & 0.00315 \\
  -0.0762 & 0.0674 & 0.0248 \\
\end{pmatrix}
\]
}

Filtro 6
{ \small
\[
\begin{pmatrix}
  -0.18 & 0.132 & -0.148 \\
  -0.332 & 0.14 & 0.0837 \\
  -0.148 & -0.0521 & 0.0463 \\
\end{pmatrix}
\begin{pmatrix}
  1.24 & 2.54 & 2.04 \\
  1.42 & 3.72 & 2.9 \\
  0.808 & 1.4 & 1.25 \\
\end{pmatrix}
\begin{pmatrix}
  0.141 & 0.0728 & -0.276 \\
  0.155 & 0.00944 & -0.482 \\
  0.0261 & -0.112 & -0.253 \\
\end{pmatrix}
\]
\[
\begin{pmatrix}
  0.0495 & 0.252 & -0.387 \\
  0.0507 & 0.263 & -0.433 \\
  -0.0262 & 0.372 & -0.173 \\
\end{pmatrix}
\begin{pmatrix}
  -0.0983 & 0.311 & -0.584 \\
  -0.0917 & 0.504 & -0.506 \\
  -0.146 & 0.283 & -0.439 \\
\end{pmatrix}
\begin{pmatrix}
  0.146 & 0.027 & -0.122 \\
  -0.00639 & 0.133 & -0.0593 \\
  0.0529 & -0.00195 & -0.188 \\
\end{pmatrix}
\]
}

Filtro 7
{ \small
\[
\begin{pmatrix}
  -0.132 & -0.197 & -0.384 \\
  0.603 & 0.515 & -0.173 \\
  0.275 & 0.0179 & -0.464 \\
\end{pmatrix}
\begin{pmatrix}
  0.257 & -0.142 & -0.441 \\
  0.818 & 0.749 & -0.163 \\
  0.59 & 0.391 & -0.341 \\
\end{pmatrix}
\begin{pmatrix}
  0.00286 & 0.0361 & 0.0179 \\
  -0.00501 & -0.0232 & 0.0253 \\
  -0.0264 & -0.0285 & -0.0251 \\
\end{pmatrix}
\]
\[
\begin{pmatrix}
  -0.235 & -0.164 & -0.128 \\
  -0.131 & 0.048 & -0.0451 \\
  0.212 & 0.216 & -0.0135 \\
\end{pmatrix}
\begin{pmatrix}
  -0.31 & -0.567 & -0.735 \\
  0.954 & 0.805 & -0.114 \\
  0.392 & 0.0886 & -0.499 \\
\end{pmatrix}
\begin{pmatrix}
  0.00732 & 0.11 & -0.149 \\
  0.179 & 0.383 & 0.0674 \\
  -0.0742 & 0.00327 & -0.103 \\
\end{pmatrix}
\]
}

Filtro 8
{ \small
\[
\begin{pmatrix}
  -0.519 & -0.772 & -0.127 \\
  0.0436 & -0.59 & -0.0953 \\
  0.759 & 0.381 & 0.554 \\
\end{pmatrix}
\begin{pmatrix}
  -0.137 & -0.00893 & 0.138 \\
  -0.357 & 0.171 & 0.712 \\
  1.36 & 1.14 & 1.54 \\
\end{pmatrix}
\begin{pmatrix}
  0.0212 & 0.0379 & 0.0148 \\
  0.0319 & 0.0375 & 0.00565 \\
  -0.0128 & -0.00937 & -0.00346 \\
\end{pmatrix}
\]
\[
\begin{pmatrix}
  -0.0323 & -0.0522 & -0.0505 \\
  -0.114 & -0.116 & -0.0864 \\
  -0.0666 & 0.0332 & -0.045 \\
\end{pmatrix}
\begin{pmatrix}
  -0.547 & -0.375 & -0.19 \\
  -0.528 & -0.573 & -0.282 \\
  0.0142 & 0.0574 & -0.0251 \\
\end{pmatrix}
\begin{pmatrix}
  -0.0748 & -0.142 & -0.0863 \\
  -0.0488 & -0.0889 & -0.129 \\
  -0.009 & 0.0101 & 0.025 \\
\end{pmatrix}
\]
}
\textbf{M3: 8 convoluções $3 \times 3$ aplicadas sobre uma entrada que 
possui 8 canais}

Filtro 1
{ \small
\[
\begin{pmatrix}
  1.42 & 0.146 & -1.1 \\
  -0.497 & -1.07 & 0.764 \\
  -0.733 & 0.421 & -0.123 \\
\end{pmatrix}
\begin{pmatrix}
  0.29 & 0.187 & 0.445 \\
  0.174 & -0.181 & 0.306 \\
  -0.118 & 0.364 & 0.156 \\
\end{pmatrix}
\begin{pmatrix}
  -0.773 & -0.0784 & -0.0103 \\
  0.366 & 0.517 & 0.462 \\
  -0.0423 & -0.402 & 0.445 \\
\end{pmatrix}
\]
\[
\begin{pmatrix}
  0.743 & -0.117 & -0.0442 \\
  -0.448 & -0.0934 & -0.188 \\
  0.258 & 0.22 & -0.434 \\
\end{pmatrix}
\begin{pmatrix}
  0.655 & 0.429 & -0.814 \\
  -0.0675 & -0.501 & 0.442 \\
  -0.269 & 0.728 & -0.64 \\
\end{pmatrix}
\begin{pmatrix}
  1.33 & -0.567 & 0.0951 \\
  -0.401 & -0.409 & -0.823 \\
  -0.726 & 0.645 & -0.639 \\
\end{pmatrix}
\]
\[
\begin{pmatrix}
  -0.548 & -0.012 & 0.478 \\
  -0.145 & 0.162 & 0.0321 \\
  -0.273 & 0.0352 & -0.284 \\
\end{pmatrix}
\begin{pmatrix}
  -0.835 & 0.0177 & 0.0969 \\
  0.458 & 0.383 & 0.22 \\
  -0.327 & -0.209 & 0.394 \\
\end{pmatrix}
\]
}

Filtro 2
{ \small
\[
\begin{pmatrix}
  0.391 & -0.367 & 1.46 \\
  -0.514 & -0.433 & -0.418 \\
  -0.0758 & 0.254 & -0.302 \\
\end{pmatrix}
\begin{pmatrix}
  -0.717 & -0.188 & -0.597 \\
  -0.381 & -0.246 & -0.241 \\
  -0.353 & -0.133 & 0.179 \\
\end{pmatrix}
\begin{pmatrix}
  -0.0892 & -1.58 & 0.849 \\
  0.115 & 0.303 & 0.222 \\
  -0.155 & 0.216 & -0.305 \\
\end{pmatrix}
\]
\[
\begin{pmatrix}
  0.455 & 0.618 & -0.0881 \\
  -0.321 & -0.252 & -0.414 \\
  0.051 & 0.000379 & -0.00894 \\
\end{pmatrix}
\begin{pmatrix}
  0.416 & 1.66 & -1.49 \\
  -0.446 & 0.288 & -1.1 \\
  0.221 & -0.0778 & 0.38 \\
\end{pmatrix}
\begin{pmatrix}
  -0.197 & -0.102 & -0.152 \\
  0.483 & 0.263 & -0.256 \\
  -0.232 & -0.0665 & -0.293 \\
\end{pmatrix}
\]
\[
\begin{pmatrix}
  0.48 & 0.753 & 0.393 \\
  -0.0545 & 0.242 & 0.307 \\
  -0.526 & -0.382 & -0.145 \\
\end{pmatrix}
\begin{pmatrix}
  -0.333 & -1.07 & 0.439 \\
  0.219 & 0.262 & 0.409 \\
  -0.091 & 0.173 & -0.0495 \\
\end{pmatrix}
\]
}

Filtro 3
{ \small
\[
\begin{pmatrix}
  -0.0957 & -0.119 & -0.105 \\
  0.37 & -1.2 & 0.206 \\
  0.424 & -0.0119 & 0.273 \\
\end{pmatrix}
\begin{pmatrix}
  0.0419 & 0.0299 & 0.03 \\
  0.158 & 0.0845 & -0.0396 \\
  0.0163 & 2.63e-06 & -0.283 \\
\end{pmatrix}
\begin{pmatrix}
  0.0748 & 0.536 & 0.0212 \\
  0.293 & -0.547 & -0.926 \\
  0.38 & 0.22 & 0.00875 \\
\end{pmatrix}
\]
\[
\begin{pmatrix}
  -0.194 & -0.154 & 0.31 \\
  -0.501 & 0.289 & 0.55 \\
  -0.446 & 0.0732 & 0.118 \\
\end{pmatrix}
\begin{pmatrix}
  0.075 & -0.21 & 0.224 \\
  -0.167 & 0.422 & -0.36 \\
  -0.0707 & 0.536 & -0.18 \\
\end{pmatrix}
\begin{pmatrix}
  -0.179 & -0.0517 & 0.12 \\
  0.759 & 0.162 & 0.626 \\
  -0.537 & 0.57 & 0.0237 \\
\end{pmatrix}
\]
\[
\begin{pmatrix}
  -0.181 & 0.0147 & -0.119 \\
  0.636 & 0.0994 & 0.142 \\
  -0.0201 & -0.588 & -0.253 \\
\end{pmatrix}
\begin{pmatrix}
  0.197 & 0.325 & -0.242 \\
  0.675 & -0.382 & -0.856 \\
  0.627 & -0.0991 & -0.259 \\
\end{pmatrix}
\]
}

Filtro 4
{ \small
\[
\begin{pmatrix}
  0.113 & -0.224 & 0.0206 \\
  -0.119 & 0.121 & 0.35 \\
  0.223 & 0.00787 & 0.0028 \\
\end{pmatrix}
\begin{pmatrix}
  0.189 & 0.0625 & 0.0535 \\
  0.137 & -0.142 & -0.169 \\
  0.0443 & -0.295 & -0.063 \\
\end{pmatrix}
\begin{pmatrix}
  -0.649 & -0.0986 & 0.754 \\
  -0.393 & 0.356 & 0.354 \\
  0.163 & -0.018 & -0.847 \\
\end{pmatrix}
\]
\[
\begin{pmatrix}
  0.0913 & 0.101 & -0.237 \\
  -0.447 & 0.273 & -0.948 \\
  0.0849 & 0.958 & 0.177 \\
\end{pmatrix}
\begin{pmatrix}
  0.13 & -0.0383 & 0.166 \\
  -0.379 & 0.237 & -0.872 \\
  0.0225 & 0.693 & 0.0437 \\
\end{pmatrix}
\begin{pmatrix}
  -0.139 & -0.162 & 0.0152 \\
  -0.206 & -0.0531 & -0.638 \\
  0.186 & -0.316 & -0.25 \\
\end{pmatrix}
\]
\[
\begin{pmatrix}
  -0.264 & -0.0338 & -0.218 \\
  0.113 & 0.607 & 0.0954 \\
  -0.172 & 0.117 & 0.144 \\
\end{pmatrix}
\begin{pmatrix}
  -0.093 & -0.109 & 0.504 \\
  0.411 & -0.258 & 1.01 \\
  -0.0291 & -1.19 & -0.231 \\
\end{pmatrix}
\]
}

Filtro 5
{ \small
\[
\begin{pmatrix}
  -0.0956 & 0.0253 & 0.208 \\
  -0.0762 & 0.115 & -0.0482 \\
  0.219 & 0.0807 & -0.0225 \\
\end{pmatrix}
\begin{pmatrix}
  0.225 & 0.119 & 0.1 \\
  -0.111 & -0.126 & 0.0196 \\
  0.00336 & 0.158 & -0.0449 \\
\end{pmatrix}
\begin{pmatrix}
  0.121 & -0.286 & -0.594 \\
  0.0114 & 0.182 & 0.124 \\
  0.419 & 0.603 & 0.415 \\
\end{pmatrix}
\]
\[
\begin{pmatrix}
  -0.337 & 0.546 & 0.901 \\
  0.0834 & 0.298 & -0.167 \\
  -0.119 & -0.495 & -0.642 \\
\end{pmatrix}
\begin{pmatrix}
  -0.116 & 0.469 & 0.548 \\
  -0.197 & 0.0623 & -0.332 \\
  0.0783 & -0.131 & -0.139 \\
\end{pmatrix}
\begin{pmatrix}
  0.289 & 0.474 & 0.634 \\
  -0.163 & -0.646 & -0.147 \\
  0.0151 & -0.132 & -0.172 \\
\end{pmatrix}
\]
\[
\begin{pmatrix}
  0.193 & 0.263 & 0.00726 \\
  -0.347 & 0.167 & -0.466 \\
  0.063 & 0.067 & 0.0908 \\
\end{pmatrix}
\begin{pmatrix}
  0.538 & -0.528 & -0.849 \\
  -0.252 & -0.569 & 0.0183 \\
  0.204 & 0.636 & 0.752 \\
\end{pmatrix}
\]
}

Filtro 6
{ \small
\[
\begin{pmatrix}
  0.464 & -0.343 & -0.261 \\
  -0.149 & 0.537 & 0.309 \\
  0.00116 & -0.0594 & -0.304 \\
\end{pmatrix}
\begin{pmatrix}
  0.148 & -0.295 & 0.0321 \\
  -0.164 & -0.0409 & -0.0357 \\
  0.192 & 0.0123 & -0.0508 \\
\end{pmatrix}
\begin{pmatrix}
  -0.743 & 0.13 & -0.225 \\
  -0.574 & 0.2 & 0.593 \\
  -0.319 & 0.231 & 0.00779 \\
\end{pmatrix}
\]
\[
\begin{pmatrix}
  0.697 & 0.19 & -0.428 \\
  0.381 & 0.116 & -0.935 \\
  0.571 & 0.0439 & -0.619 \\
\end{pmatrix}
\begin{pmatrix}
  0.268 & -0.45 & 0.18 \\
  0.21 & -0.0314 & 0.0229 \\
  0.4 & -0.365 & -0.164 \\
\end{pmatrix}
\begin{pmatrix}
  0.246 & -0.31 & 0.135 \\
  0.692 & -0.421 & -0.886 \\
  0.512 & -0.68 & 0.35 \\
\end{pmatrix}
\]
\[
\begin{pmatrix}
  0.805 & -0.00815 & -0.119 \\
  -0.0886 & -0.101 & -0.357 \\
  0.26 & -0.504 & 0.102 \\
\end{pmatrix}
\begin{pmatrix}
  -0.752 & -0.228 & 0.498 \\
  -0.491 & -0.172 & 1.05 \\
  -0.535 & -0.177 & 0.771 \\
\end{pmatrix}
\]
}

Filtro 7
{ \small
\[
\begin{pmatrix}
  0.198 & 0.173 & 0.516 \\
  0.428 & -0.632 & -0.255 \\
  0.115 & -0.632 & -0.089 \\
\end{pmatrix}
\begin{pmatrix}
  0.101 & -0.116 & -0.237 \\
  -0.0516 & -0.219 & 0.0321 \\
  0.0814 & -0.152 & -0.00684 \\
\end{pmatrix}
\begin{pmatrix}
  -0.152 & 0.0487 & -0.557 \\
  0.834 & -0.013 & -0.423 \\
  0.406 & 0.0459 & 0.0957 \\
\end{pmatrix}
\]
\[
\begin{pmatrix}
  -0.0621 & 0.0841 & 0.718 \\
  -0.816 & -0.543 & 0.584 \\
  -0.161 & -0.246 & 0.487 \\
\end{pmatrix}
\begin{pmatrix}
  0.0694 & 0.265 & 0.209 \\
  -0.0523 & -0.31 & 0.248 \\
  -0.303 & -0.269 & 0.277 \\
\end{pmatrix}
\begin{pmatrix}
  0.245 & 0.35 & 0.158 \\
  -0.784 & -0.092 & -0.0821 \\
  -0.144 & -0.0478 & 0.249 \\
\end{pmatrix}
\]
\[
\begin{pmatrix}
  0.17 & -0.111 & 0.176 \\
  -0.0662 & -0.944 & 0.194 \\
  0.0828 & -0.4 & 0.0766 \\
\end{pmatrix}
\begin{pmatrix}
  0.214 & -0.0778 & -0.733 \\
  1.05 & 0.504 & -0.87 \\
  0.168 & 0.225 & -0.412 \\
\end{pmatrix}
\]
}

Filtro 8
{ \small
\[
\begin{pmatrix}
  0.236 & 0.323 & -0.298 \\
  0.172 & 0.218 & -0.312 \\
  -0.705 & -0.559 & 0.376 \\
\end{pmatrix}
\begin{pmatrix}
  0.191 & 0.189 & 0.0437 \\
  0.0764 & 0.333 & -0.0474 \\
  -0.175 & 0.281 & 0.329 \\
\end{pmatrix}
\begin{pmatrix}
  0.196 & 0.199 & -0.606 \\
  0.282 & 0.736 & 0.491 \\
  0.0534 & -0.754 & -0.0951 \\
\end{pmatrix}
\]
\[
\begin{pmatrix}
  0.194 & 0.0159 & -0.244 \\
  -0.677 & -0.788 & -0.44 \\
  0.754 & 0.588 & 0.569 \\
\end{pmatrix}
\begin{pmatrix}
  0.515 & -0.5 & -0.0791 \\
  0.0612 & 0.000762 & -0.0175 \\
  -0.3 & -0.00435 & 0.00866 \\
\end{pmatrix}
\begin{pmatrix}
  0.124 & -0.263 & 0.223 \\
  -0.184 & -0.454 & -0.208 \\
  0.269 & 0.427 & 0.39 \\
\end{pmatrix}
\]
\[
\begin{pmatrix}
  -0.162 & -0.0127 & -0.193 \\
  0.154 & 0.443 & -0.139 \\
  0.116 & 0.193 & 0.17 \\
\end{pmatrix}
\begin{pmatrix}
  -0.0787 & 0.00762 & 0.0229 \\
  0.621 & 0.992 & 0.465 \\
  -0.724 & -0.606 & -0.521 \\
\end{pmatrix}
\]
}
\textbf{M4: 8 convoluções $3 \times 3$ aplicadas sobre uma entrada que 
possui 8 canais}

Filtro 1
{ \small
\[
\begin{pmatrix}
  -0.126 & 0.318 & 0.282 \\
  -0.426 & 0.254 & 0.274 \\
  0.736 & -0.677 & -0.116 \\
\end{pmatrix}
\begin{pmatrix}
  0.275 & 0.0883 & 0.327 \\
  -0.0733 & 0.278 & -0.139 \\
  -0.045 & -0.649 & 0.0208 \\
\end{pmatrix}
\begin{pmatrix}
  0.237 & -0.232 & 0.0825 \\
  -0.0801 & -0.862 & 0.257 \\
  0.0552 & -0.119 & -0.226 \\
\end{pmatrix}
\]
\[
\begin{pmatrix}
  0.348 & 0.366 & -0.407 \\
  -0.5 & 0.211 & 0.402 \\
  -0.211 & -0.086 & 0.0245 \\
\end{pmatrix}
\begin{pmatrix}
  0.471 & -0.342 & -0.144 \\
  -0.416 & 0.427 & 0.104 \\
  -0.341 & 0.349 & 0.098 \\
\end{pmatrix}
\begin{pmatrix}
  0.0965 & 0.0733 & 0.0868 \\
  -0.0826 & 0.408 & 0.132 \\
  0.2 & -0.0555 & 0.00453 \\
\end{pmatrix}
\]
\[
\begin{pmatrix}
  -0.276 & 0.0646 & -0.0216 \\
  0.348 & -0.257 & 0.169 \\
  -0.232 & -0.62 & 0.255 \\
\end{pmatrix}
\begin{pmatrix}
  -0.421 & -0.61 & -0.189 \\
  -0.642 & 0.138 & 0.298 \\
  0.288 & 0.15 & -0.406 \\
\end{pmatrix}
\]
}

Filtro 2
{ \small
\[
\begin{pmatrix}
  -0.0381 & 0.171 & -0.153 \\
  0.186 & -0.653 & -0.351 \\
  -0.432 & 0.304 & -0.177 \\
\end{pmatrix}
\begin{pmatrix}
  0.192 & -0.0516 & 0.174 \\
  0.719 & 0.124 & 0.686 \\
  0.104 & -0.489 & -0.554 \\
\end{pmatrix}
\begin{pmatrix}
  -0.123 & -0.0463 & 0.289 \\
  -0.342 & -0.552 & -0.304 \\
  0.164 & 0.783 & 0.0861 \\
\end{pmatrix}
\]
\[
\begin{pmatrix}
  0.344 & 0.126 & -0.403 \\
  0.597 & -0.14 & 0.4 \\
  -0.414 & -0.151 & 0.287 \\
\end{pmatrix}
\begin{pmatrix}
  -0.0765 & -0.111 & 0.272 \\
  0.37 & -0.469 & -0.349 \\
  -0.0287 & -0.0765 & -0.0177 \\
\end{pmatrix}
\begin{pmatrix}
  0.196 & 0.226 & -0.127 \\
  -0.125 & -0.161 & 1.29 \\
  -1.0 & -1.22 & -1.28 \\
\end{pmatrix}
\]
\[
\begin{pmatrix}
  0.167 & -0.75 & -0.17 \\
  0.093 & -0.478 & 0.112 \\
  0.0193 & 0.872 & 0.584 \\
\end{pmatrix}
\begin{pmatrix}
  0.619 & 0.163 & -0.0365 \\
  0.0618 & -0.486 & 0.679 \\
  0.0099 & -0.212 & 0.306 \\
\end{pmatrix}
\]
}

Filtro 3
{ \small
\[
\begin{pmatrix}
  0.04 & 0.215 & -0.831 \\
  -0.387 & -0.0526 & 0.382 \\
  0.45 & 0.105 & 0.0657 \\
\end{pmatrix}
\begin{pmatrix}
  0.601 & 0.444 & 0.195 \\
  0.0813 & 0.199 & 0.193 \\
  0.00704 & 0.0415 & -0.303 \\
\end{pmatrix}
\begin{pmatrix}
  0.13 & 0.302 & 0.459 \\
  -0.317 & -0.965 & -0.576 \\
  -0.156 & -0.157 & 0.298 \\
\end{pmatrix}
\]
\[
\begin{pmatrix}
  -0.735 & -0.133 & 0.09 \\
  -0.0654 & -0.69 & -0.358 \\
  0.537 & -0.0119 & 0.381 \\
\end{pmatrix}
\begin{pmatrix}
  0.00739 & 0.326 & 0.121 \\
  -0.483 & 0.11 & 0.817 \\
  0.165 & 0.00266 & -0.0847 \\
\end{pmatrix}
\begin{pmatrix}
  -0.27 & -0.763 & -0.198 \\
  0.144 & 0.135 & 0.253 \\
  0.0432 & 0.594 & -0.266 \\
\end{pmatrix}
\]
\[
\begin{pmatrix}
  -0.244 & -0.0823 & -0.293 \\
  0.294 & 0.478 & -0.575 \\
  0.57 & 0.446 & 0.562 \\
\end{pmatrix}
\begin{pmatrix}
  -0.64 & -0.656 & 0.136 \\
  0.428 & 0.215 & -0.286 \\
  -0.0616 & 0.23 & -0.864 \\
\end{pmatrix}
\]
}

Filtro 4
{ \small
\[
\begin{pmatrix}
  0.174 & 0.566 & -0.0368 \\
  0.0835 & -0.288 & 0.164 \\
  0.674 & 0.28 & -0.613 \\
\end{pmatrix}
\begin{pmatrix}
  -0.197 & 0.0712 & 0.0498 \\
  -0.113 & -0.0225 & -0.356 \\
  -0.833 & -0.991 & -0.796 \\
\end{pmatrix}
\begin{pmatrix}
  -0.411 & 0.0127 & -0.152 \\
  -0.205 & -0.0567 & -0.0418 \\
  0.179 & 0.0327 & 0.0164 \\
\end{pmatrix}
\]
\[
\begin{pmatrix}
  0.587 & -0.439 & -0.0613 \\
  -0.291 & -0.0677 & 0.806 \\
  -0.0572 & 0.499 & -0.132 \\
\end{pmatrix}
\begin{pmatrix}
  -0.369 & -0.247 & 0.59 \\
  -0.126 & -0.0503 & 0.422 \\
  -0.603 & 0.492 & 0.17 \\
\end{pmatrix}
\begin{pmatrix}
  -0.149 & -0.268 & 0.233 \\
  -0.155 & -0.298 & -0.795 \\
  0.319 & 0.00575 & -0.869 \\
\end{pmatrix}
\]
\[
\begin{pmatrix}
  -0.322 & 0.0616 & 0.404 \\
  0.361 & -0.134 & -0.265 \\
  0.912 & -0.437 & -0.566 \\
\end{pmatrix}
\begin{pmatrix}
  0.505 & -0.371 & -0.449 \\
  0.182 & -0.736 & -0.418 \\
  0.102 & -0.0962 & -0.124 \\
\end{pmatrix}
\]
}

Filtro 5
{ \small
\[
\begin{pmatrix}
  -0.155 & -0.0132 & 0.163 \\
  -0.159 & 0.221 & 0.364 \\
  0.2 & 0.168 & -0.17 \\
\end{pmatrix}
\begin{pmatrix}
  -0.0457 & -0.0117 & -0.312 \\
  -0.447 & 0.0764 & -0.24 \\
  0.113 & -0.411 & 0.108 \\
\end{pmatrix}
\begin{pmatrix}
  -0.0427 & 0.629 & -0.235 \\
  -0.0439 & -0.15 & -0.555 \\
  0.0717 & -0.154 & -0.079 \\
\end{pmatrix}
\]
\[
\begin{pmatrix}
  0.663 & -0.00833 & -0.849 \\
  0.111 & 0.349 & -0.252 \\
  0.0576 & -0.32 & 0.524 \\
\end{pmatrix}
\begin{pmatrix}
  -0.516 & 0.345 & 0.383 \\
  -0.0467 & 0.494 & -0.362 \\
  0.502 & 0.229 & -0.732 \\
\end{pmatrix}
\begin{pmatrix}
  -0.203 & -0.031 & 0.415 \\
  -0.304 & -0.0385 & 0.676 \\
  0.605 & 0.312 & 0.13 \\
\end{pmatrix}
\]
\[
\begin{pmatrix}
  0.334 & 0.157 & 0.0389 \\
  -0.345 & -0.121 & 0.292 \\
  -0.259 & -0.485 & 0.376 \\
\end{pmatrix}
\begin{pmatrix}
  1.08 & -1.1 & -0.212 \\
  0.586 & -1.11 & -0.223 \\
  -0.534 & -0.00265 & 0.37 \\
\end{pmatrix}
\]
}

Filtro 6
{ \small
\[
\begin{pmatrix}
  0.741 & 0.383 & -0.0657 \\
  -0.506 & 0.107 & -0.0759 \\
  -0.0306 & -0.221 & -0.152 \\
\end{pmatrix}
\begin{pmatrix}
  -0.294 & 0.174 & -0.333 \\
  0.571 & 0.221 & -0.51 \\
  0.549 & 0.285 & 0.167 \\
\end{pmatrix}
\begin{pmatrix}
  0.122 & -0.332 & -0.571 \\
  0.859 & -0.0807 & -0.163 \\
  0.113 & -0.173 & -0.22 \\
\end{pmatrix}
\]
\[
\begin{pmatrix}
  0.957 & 0.0174 & -0.128 \\
  -0.207 & -0.281 & -0.0835 \\
  0.204 & -0.325 & 0.458 \\
\end{pmatrix}
\begin{pmatrix}
  -0.75 & -0.153 & 0.192 \\
  -0.223 & 0.467 & -0.144 \\
  -0.168 & 0.586 & -0.251 \\
\end{pmatrix}
\begin{pmatrix}
  0.243 & 0.155 & 0.258 \\
  -0.48 & -0.443 & 0.215 \\
  0.405 & -0.367 & 0.512 \\
\end{pmatrix}
\]
\[
\begin{pmatrix}
  0.292 & 0.259 & 0.0625 \\
  -1.11 & -0.329 & -0.0615 \\
  -0.308 & 0.27 & 0.171 \\
\end{pmatrix}
\begin{pmatrix}
  0.635 & 0.0179 & -0.492 \\
  0.00209 & -0.0356 & 0.0182 \\
  -0.191 & 0.116 & 0.109 \\
\end{pmatrix}
\]
}

Filtro 7
{ \small
\[
\begin{pmatrix}
  0.48 & -0.403 & 0.558 \\
  -0.64 & 0.0221 & -0.113 \\
  -0.666 & 0.0961 & -0.04 \\
\end{pmatrix}
\begin{pmatrix}
  -0.462 & -0.0988 & 0.269 \\
  0.462 & -0.736 & -0.0463 \\
  -0.196 & 0.108 & -0.0051 \\
\end{pmatrix}
\begin{pmatrix}
  -0.0152 & -0.169 & -0.0013 \\
  -0.0641 & 0.0643 & 0.075 \\
  -0.299 & 0.16 & 0.131 \\
\end{pmatrix}
\]
\[
\begin{pmatrix}
  1.32 & -1.14 & 0.258 \\
  0.26 & -0.714 & 0.208 \\
  0.0221 & 0.194 & -0.473 \\
\end{pmatrix}
\begin{pmatrix}
  -0.268 & -0.384 & 0.298 \\
  -0.249 & -0.35 & 0.607 \\
  -0.775 & -0.0426 & 0.783 \\
\end{pmatrix}
\begin{pmatrix}
  -0.131 & -0.227 & -0.319 \\
  0.433 & -0.372 & -0.705 \\
  -0.23 & 0.466 & -0.198 \\
\end{pmatrix}
\]
\[
\begin{pmatrix}
  0.299 & 0.102 & 0.354 \\
  -0.489 & 0.278 & 0.839 \\
  0.292 & 0.00108 & -0.0399 \\
\end{pmatrix}
\begin{pmatrix}
  0.102 & -0.889 & -0.469 \\
  0.103 & 0.206 & 0.225 \\
  0.161 & 0.556 & -0.246 \\
\end{pmatrix}
\]
}

Filtro 8
{ \small
\[
\begin{pmatrix}
  -0.337 & 0.206 & 0.269 \\
  -0.525 & -0.236 & 0.724 \\
  -1.17 & -1.37 & -0.778 \\
\end{pmatrix}
\begin{pmatrix}
  -0.0487 & 0.135 & 0.044 \\
  0.0904 & 0.479 & 0.475 \\
  0.612 & 0.021 & -0.115 \\
\end{pmatrix}
\begin{pmatrix}
  -0.15 & 0.107 & -0.0648 \\
  -0.266 & 0.288 & 0.409 \\
  -0.0241 & 0.362 & -0.313 \\
\end{pmatrix}
\]
\[
\begin{pmatrix}
  0.495 & 0.102 & 0.249 \\
  -0.0719 & -0.0695 & 0.116 \\
  0.0998 & 0.633 & -0.25 \\
\end{pmatrix}
\begin{pmatrix}
  -0.429 & -0.0752 & 0.136 \\
  -1.18 & -0.552 & -0.213 \\
  -0.961 & -0.122 & 0.942 \\
\end{pmatrix}
\begin{pmatrix}
  -0.0144 & -0.192 & 0.576 \\
  0.86 & -0.417 & -0.34 \\
  0.249 & 0.341 & -1.08 \\
\end{pmatrix}
\]
\[
\begin{pmatrix}
  -0.252 & 0.113 & 0.493 \\
  0.000428 & -0.557 & 0.163 \\
  1.02 & -0.1 & -0.252 \\
\end{pmatrix}
\begin{pmatrix}
  -0.454 & -0.305 & 0.439 \\
  0.224 & -0.00513 & -1.38 \\
  0.126 & -0.546 & -0.0965 \\
\end{pmatrix}
\]
}
\textbf{M5: 8 convoluções $3 \times 3$ aplicadas sobre uma entrada que 
possui 8 canais}

Filtro 1
{ \small
\[
\begin{pmatrix}
  -0.184 & -0.114 & -0.168 \\
  -0.131 & -0.113 & 0.127 \\
  0.0721 & -0.0332 & 0.0568 \\
\end{pmatrix}
\begin{pmatrix}
  0.306 & -0.117 & -0.177 \\
  0.0808 & 0.205 & -0.145 \\
  0.268 & -0.0537 & 0.126 \\
\end{pmatrix}
\begin{pmatrix}
  0.393 & -0.341 & -0.538 \\
  0.391 & -0.224 & -0.176 \\
  0.35 & 0.217 & -0.189 \\
\end{pmatrix}
\]
\[
\begin{pmatrix}
  -0.0507 & 0.281 & 0.0462 \\
  0.101 & -0.0151 & -0.0781 \\
  -0.334 & 0.0774 & -0.25 \\
\end{pmatrix}
\begin{pmatrix}
  -0.237 & -0.278 & -0.464 \\
  0.712 & 0.455 & 0.131 \\
  0.352 & -0.115 & -0.0735 \\
\end{pmatrix}
\begin{pmatrix}
  0.402 & -0.0893 & -0.632 \\
  -0.00046 & -0.243 & 0.268 \\
  0.808 & -0.125 & -0.0826 \\
\end{pmatrix}
\]
\[
\begin{pmatrix}
  0.0388 & 0.0949 & 0.0506 \\
  0.0852 & -0.181 & 0.374 \\
  -0.0493 & 0.0915 & -0.514 \\
\end{pmatrix}
\begin{pmatrix}
  0.779 & -0.275 & -0.269 \\
  0.0786 & -0.1 & 0.235 \\
  -0.509 & 0.285 & -0.469 \\
\end{pmatrix}
\]
}

Filtro 2
{ \small
\[
\begin{pmatrix}
  -0.212 & 0.194 & -0.211 \\
  -0.0553 & 0.115 & 0.618 \\
  0.366 & 0.366 & -0.443 \\
\end{pmatrix}
\begin{pmatrix}
  -0.26 & 0.389 & 0.518 \\
  -0.142 & 0.0653 & -0.0458 \\
  -0.479 & 0.17 & -0.499 \\
\end{pmatrix}
\begin{pmatrix}
  -0.223 & -0.218 & -0.00793 \\
  0.121 & -0.308 & -0.999 \\
  -0.811 & -0.402 & 0.182 \\
\end{pmatrix}
\]
\[
\begin{pmatrix}
  -0.579 & 0.208 & -0.168 \\
  -0.181 & -0.0571 & 0.343 \\
  -0.186 & 0.233 & 0.634 \\
\end{pmatrix}
\begin{pmatrix}
  0.36 & -0.456 & 0.266 \\
  -0.189 & 0.559 & 0.51 \\
  0.141 & 0.29 & 0.307 \\
\end{pmatrix}
\begin{pmatrix}
  -0.634 & 0.443 & 0.0187 \\
  0.0252 & -0.486 & 0.107 \\
  -0.661 & -0.118 & 0.0668 \\
\end{pmatrix}
\]
\[
\begin{pmatrix}
  0.702 & 0.426 & 0.149 \\
  0.329 & -0.0261 & 0.319 \\
  -1.03 & -0.0508 & -0.465 \\
\end{pmatrix}
\begin{pmatrix}
  -0.258 & -0.0431 & 0.0825 \\
  -0.434 & -0.192 & 0.32 \\
  0.12 & 0.0763 & 0.439 \\
\end{pmatrix}
\]
}

Filtro 3
{ \small
\[
\begin{pmatrix}
  -0.112 & 0.248 & 0.521 \\
  -0.189 & -0.00982 & 0.334 \\
  -0.0185 & 0.0286 & 0.386 \\
\end{pmatrix}
\begin{pmatrix}
  -0.0503 & -0.134 & -0.326 \\
  -0.35 & -0.122 & -0.824 \\
  0.0233 & 0.0669 & 0.00824 \\
\end{pmatrix}
\begin{pmatrix}
  -0.394 & 0.296 & -0.104 \\
  -0.231 & 0.384 & -0.821 \\
  -0.103 & 0.0951 & 0.00772 \\
\end{pmatrix}
\]
\[
\begin{pmatrix}
  -0.0468 & -0.151 & 0.205 \\
  -0.447 & -0.183 & -0.677 \\
  -0.101 & 0.0318 & 0.016 \\
\end{pmatrix}
\begin{pmatrix}
  0.0935 & -0.273 & -0.0736 \\
  0.277 & -0.867 & 0.532 \\
  0.00606 & -0.388 & -0.091 \\
\end{pmatrix}
\begin{pmatrix}
  -0.284 & 0.087 & 0.00516 \\
  0.0307 & 0.302 & -0.711 \\
  0.242 & 0.127 & -0.111 \\
\end{pmatrix}
\]
\[
\begin{pmatrix}
  -0.264 & -0.348 & -0.155 \\
  0.124 & 0.564 & 0.157 \\
  0.28 & 0.658 & -0.81 \\
\end{pmatrix}
\begin{pmatrix}
  0.405 & 0.109 & 0.275 \\
  0.176 & -0.0725 & -0.0528 \\
  -0.247 & -0.099 & -0.134 \\
\end{pmatrix}
\]
}

Filtro 4
{ \small
\[
\begin{pmatrix}
  0.18 & 0.166 & 0.327 \\
  -0.0425 & -0.187 & 0.218 \\
  0.166 & 0.0453 & -0.0177 \\
\end{pmatrix}
\begin{pmatrix}
  -0.372 & -0.275 & -0.401 \\
  -0.402 & -0.168 & -0.492 \\
  -0.173 & 0.0232 & 0.124 \\
\end{pmatrix}
\begin{pmatrix}
  0.0724 & -0.498 & -0.557 \\
  -0.166 & -0.608 & -0.575 \\
  -0.23 & -0.187 & -0.502 \\
\end{pmatrix}
\]
\[
\begin{pmatrix}
  -0.28 & -0.191 & -0.491 \\
  -0.284 & 0.19 & -0.28 \\
  0.128 & -0.122 & 0.225 \\
\end{pmatrix}
\begin{pmatrix}
  -0.043 & 0.00429 & 0.28 \\
  0.334 & 0.371 & 0.318 \\
  -0.021 & 0.0102 & 0.463 \\
\end{pmatrix}
\begin{pmatrix}
  0.0591 & 0.155 & 0.259 \\
  -0.0135 & 0.0679 & -0.0888 \\
  0.0612 & 0.238 & 0.168 \\
\end{pmatrix}
\]
\[
\begin{pmatrix}
  -0.31 & -0.25 & 0.39 \\
  0.00431 & -0.203 & 0.0925 \\
  0.0379 & 0.0933 & 0.109 \\
\end{pmatrix}
\begin{pmatrix}
  -0.161 & -0.681 & -0.629 \\
  0.141 & 0.317 & -0.406 \\
  -1.21 & 0.0487 & -0.47 \\
\end{pmatrix}
\]
}

Filtro 5
{ \small
\[
\begin{pmatrix}
  0.0945 & 0.256 & -0.309 \\
  -0.439 & -0.325 & -0.456 \\
  0.471 & 0.113 & -0.23 \\
\end{pmatrix}
\begin{pmatrix}
  -0.31 & -0.061 & 0.602 \\
  0.124 & -0.677 & 0.03 \\
  -0.28 & -1.06 & -0.101 \\
\end{pmatrix}
\begin{pmatrix}
  -0.0954 & 0.221 & -0.08 \\
  -0.502 & 0.323 & -0.234 \\
  -0.435 & 0.186 & 0.015 \\
\end{pmatrix}
\]
\[
\begin{pmatrix}
  0.324 & 0.103 & 0.296 \\
  -0.202 & 0.186 & 0.582 \\
  -0.274 & -0.575 & 0.404 \\
\end{pmatrix}
\begin{pmatrix}
  -0.241 & -0.714 & -0.527 \\
  0.842 & 0.405 & -0.173 \\
  0.151 & 0.351 & -0.0172 \\
\end{pmatrix}
\begin{pmatrix}
  0.112 & -0.141 & -0.0953 \\
  0.157 & -0.18 & -1.5 \\
  0.28 & 0.229 & -0.629 \\
\end{pmatrix}
\]
\[
\begin{pmatrix}
  -0.383 & -0.44 & -0.631 \\
  -0.45 & 0.0843 & -0.0358 \\
  -0.106 & 0.594 & 0.321 \\
\end{pmatrix}
\begin{pmatrix}
  0.21 & -0.344 & 0.301 \\
  0.298 & -0.917 & -1.14 \\
  0.482 & -0.36 & -0.481 \\
\end{pmatrix}
\]
}

Filtro 6
{ \small
\[
\begin{pmatrix}
  -0.502 & 0.0983 & -0.224 \\
  -0.0431 & 0.25 & -0.042 \\
  -0.298 & -0.158 & -0.0142 \\
\end{pmatrix}
\begin{pmatrix}
  -0.128 & -0.104 & -0.344 \\
  0.253 & 0.0971 & -0.163 \\
  0.332 & -0.251 & 0.132 \\
\end{pmatrix}
\begin{pmatrix}
  -0.109 & 0.0735 & 0.161 \\
  -0.254 & 0.291 & -0.0366 \\
  -1.61 & -0.466 & -0.0115 \\
\end{pmatrix}
\]
\[
\begin{pmatrix}
  -0.166 & 0.0383 & -0.00769 \\
  0.711 & 0.0472 & 0.385 \\
  0.167 & -0.151 & 0.247 \\
\end{pmatrix}
\begin{pmatrix}
  -0.469 & 0.115 & 0.304 \\
  -0.642 & 0.246 & 0.566 \\
  -0.0232 & -0.206 & 0.00396 \\
\end{pmatrix}
\begin{pmatrix}
  -0.445 & -0.132 & 0.339 \\
  -1.41 & -0.585 & -0.879 \\
  -0.0578 & 0.0705 & 0.0357 \\
\end{pmatrix}
\]
\[
\begin{pmatrix}
  0.12 & -0.242 & -0.208 \\
  -0.328 & -0.482 & -0.0644 \\
  -0.234 & 0.242 & -0.108 \\
\end{pmatrix}
\begin{pmatrix}
  0.0733 & 0.349 & 0.425 \\
  0.98 & 1.07 & -0.514 \\
  0.422 & -0.121 & -0.539 \\
\end{pmatrix}
\]
}

Filtro 7
{ \small
\[
\begin{pmatrix}
  -0.0996 & 0.189 & -0.293 \\
  0.0672 & 0.668 & -0.772 \\
  0.162 & 1.04 & -1.02 \\
\end{pmatrix}
\begin{pmatrix}
  0.0901 & -0.0445 & -0.288 \\
  -0.102 & -0.245 & -0.194 \\
  -0.213 & -0.0466 & -0.0424 \\
\end{pmatrix}
\begin{pmatrix}
  -0.196 & 0.111 & 0.169 \\
  -0.0153 & 0.408 & -0.428 \\
  0.0255 & -0.184 & -0.691 \\
\end{pmatrix}
\]
\[
\begin{pmatrix}
  0.0746 & 0.164 & -0.27 \\
  0.12 & 0.564 & -0.776 \\
  -0.31 & -0.342 & 0.0481 \\
\end{pmatrix}
\begin{pmatrix}
  0.0785 & 0.0798 & 0.0191 \\
  -0.287 & -0.153 & -0.296 \\
  -0.0865 & 0.204 & 0.376 \\
\end{pmatrix}
\begin{pmatrix}
  0.00106 & -0.165 & 0.0864 \\
  -0.163 & 0.0778 & -0.0118 \\
  0.165 & 0.196 & -0.275 \\
\end{pmatrix}
\]
\[
\begin{pmatrix}
  0.0126 & -0.246 & -0.175 \\
  -0.35 & -0.24 & -0.294 \\
  0.351 & 0.349 & 0.0927 \\
\end{pmatrix}
\begin{pmatrix}
  0.0135 & -0.0214 & -0.181 \\
  0.0278 & 0.188 & 0.0828 \\
  -0.0877 & -0.313 & -0.0886 \\
\end{pmatrix}
\]
}

Filtro 8
{ \small
\[
\begin{pmatrix}
  0.0807 & -0.165 & -0.204 \\
  -0.221 & -0.393 & 0.207 \\
  -0.0244 & -0.362 & -0.126 \\
\end{pmatrix}
\begin{pmatrix}
  0.283 & 0.0141 & -0.241 \\
  0.149 & -0.315 & -0.168 \\
  -0.157 & 0.179 & -0.401 \\
\end{pmatrix}
\begin{pmatrix}
  0.0906 & -0.16 & -0.3 \\
  -0.0457 & 0.00775 & -0.301 \\
  0.0917 & 0.156 & -0.312 \\
\end{pmatrix}
\]
\[
\begin{pmatrix}
  0.174 & 0.219 & 0.0343 \\
  -0.13 & 0.463 & -0.221 \\
  0.169 & 0.396 & 0.268 \\
\end{pmatrix}
\begin{pmatrix}
  0.0888 & -0.074 & -0.361 \\
  -0.0955 & -0.332 & -0.48 \\
  0.217 & 0.277 & -0.0609 \\
\end{pmatrix}
\begin{pmatrix}
  -0.771 & -0.348 & 0.122 \\
  -0.283 & -0.631 & -0.0461 \\
  -0.287 & -0.435 & 0.2 \\
\end{pmatrix}
\]
\[
\begin{pmatrix}
  0.39 & 0.345 & -0.00402 \\
  0.39 & 0.372 & 0.219 \\
  0.225 & 0.202 & 0.028 \\
\end{pmatrix}
\begin{pmatrix}
  0.033 & 0.0182 & 0.0501 \\
  0.443 & -0.241 & -0.431 \\
  0.436 & -1.1 & -0.134 \\
\end{pmatrix}
\]
}

\textbf{M6: 6 convoluções $3 \times 3$ aplicadas sobre uma entrada que 
possui 8 canais}

Filtro 1
{ \small
\[
\begin{pmatrix}
  0.251 & -0.256 & -0.0861 \\
  -0.109 & -0.222 & -0.456 \\
  0.0623 & -0.083 & 0.108 \\
\end{pmatrix}
\begin{pmatrix}
  -0.272 & 0.376 & 0.251 \\
  -0.249 & 0.0714 & -0.176 \\
  0.203 & -0.0238 & -0.164 \\
\end{pmatrix}
\begin{pmatrix}
  -0.225 & -0.133 & -0.271 \\
  0.00288 & 0.07 & 0.313 \\
  -0.155 & -0.216 & 0.084 \\
\end{pmatrix}
\]
\[
\begin{pmatrix}
  -0.669 & -0.779 & 0.418 \\
  -0.09 & -0.202 & -0.28 \\
  0.21 & -0.104 & -0.419 \\
\end{pmatrix}
\begin{pmatrix}
  0.854 & 0.149 & -0.15 \\
  0.492 & 0.333 & 0.0754 \\
  -0.533 & 0.299 & 0.00184 \\
\end{pmatrix}
\begin{pmatrix}
  -0.245 & 0.398 & 0.0967 \\
  0.779 & 0.173 & -0.0832 \\
  0.0943 & 0.304 & -0.33 \\
\end{pmatrix}
\]
\[
\begin{pmatrix}
  0.183 & -0.127 & 0.258 \\
  0.0283 & 0.0429 & 0.185 \\
  -0.0468 & -0.0278 & -0.0338 \\
\end{pmatrix}
\begin{pmatrix}
  -0.489 & -0.431 & -0.062 \\
  -0.0337 & -0.504 & 0.00167 \\
  -0.351 & -0.231 & 0.0857 \\
\end{pmatrix}
\]
}

Filtro 2
{ \small
\[
\begin{pmatrix}
  0.011 & 0.115 & -0.0861 \\
  -0.00458 & 0.0768 & 0.125 \\
  0.0505 & 0.143 & 0.149 \\
\end{pmatrix}
\begin{pmatrix}
  0.198 & 0.13 & -0.0341 \\
  -0.085 & 0.0331 & -0.0616 \\
  -0.105 & -0.115 & -0.0275 \\
\end{pmatrix}
\begin{pmatrix}
  -0.0853 & -0.103 & -0.0799 \\
  0.0824 & -0.0444 & 0.055 \\
  0.106 & 0.107 & -0.0478 \\
\end{pmatrix}
\]
\[
\begin{pmatrix}
  -0.118 & -0.0976 & -0.332 \\
  -0.0968 & -0.163 & 0.291 \\
  -0.171 & -0.0774 & 0.0743 \\
\end{pmatrix}
\begin{pmatrix}
  0.00955 & -0.154 & 0.00169 \\
  -0.362 & -0.266 & 0.029 \\
  -0.51 & -0.235 & -0.0183 \\
\end{pmatrix}
\begin{pmatrix}
  0.01 & -0.0442 & -0.019 \\
  0.106 & -0.125 & -0.0456 \\
  0.0376 & 0.131 & -0.265 \\
\end{pmatrix}
\]
\[
\begin{pmatrix}
  0.0372 & -0.0813 & 0.055 \\
  0.0135 & -0.04 & -0.0362 \\
  0.0863 & 0.246 & -0.0849 \\
\end{pmatrix}
\begin{pmatrix}
  -0.0259 & -0.115 & -0.251 \\
  0.227 & -0.146 & -0.17 \\
  -0.424 & -0.419 & -0.313 \\
\end{pmatrix}
\]
}

Filtro 3
{ \small
\[
\begin{pmatrix}
  -0.204 & -0.189 & -0.326 \\
  0.0987 & -0.12 & -0.0584 \\
  0.0824 & -0.132 & 0.149 \\
\end{pmatrix}
\begin{pmatrix}
  -0.113 & -0.0102 & -0.361 \\
  0.176 & -0.0979 & 0.0811 \\
  -0.0171 & 0.0492 & 0.274 \\
\end{pmatrix}
\begin{pmatrix}
  0.0878 & 0.0852 & 0.0521 \\
  -0.00188 & -0.069 & 0.168 \\
  0.23 & 0.191 & 0.209 \\
\end{pmatrix}
\]
\[
\begin{pmatrix}
  -0.357 & -0.346 & 0.142 \\
  -0.319 & -0.236 & 0.0601 \\
  0.379 & -0.6 & 0.34 \\
\end{pmatrix}
\begin{pmatrix}
  0.0215 & -0.154 & 0.0468 \\
  0.197 & -0.03 & -0.0203 \\
  0.292 & -0.0481 & -0.0773 \\
\end{pmatrix}
\begin{pmatrix}
  0.0324 & 0.0845 & -0.18 \\
  0.246 & 0.412 & -0.299 \\
  0.0385 & -0.0915 & 0.00194 \\
\end{pmatrix}
\]
\[
\begin{pmatrix}
  -0.115 & -0.11 & 0.0176 \\
  0.224 & 0.0795 & 0.16 \\
  0.38 & 0.229 & 0.0477 \\
\end{pmatrix}
\begin{pmatrix}
  -0.246 & 0.0775 & 0.198 \\
  -0.297 & -0.243 & 0.133 \\
  -0.0227 & 0.0464 & -0.0382 \\
\end{pmatrix}
\]
}

Filtro 4
{ \small
\[
\begin{pmatrix}
  0.0661 & -0.402 & -0.109 \\
  -0.223 & -0.125 & -0.221 \\
  0.0803 & 0.238 & 0.465 \\
\end{pmatrix}
\begin{pmatrix}
  0.268 & 0.259 & 0.508 \\
  0.0747 & 0.108 & 0.105 \\
  -0.0397 & 0.107 & 0.121 \\
\end{pmatrix}
\begin{pmatrix}
  0.314 & -0.0318 & 0.0664 \\
  -0.15 & -0.0709 & -0.108 \\
  -0.166 & 0.0916 & -0.0157 \\
\end{pmatrix}
\]
\[
\begin{pmatrix}
  0.271 & 0.127 & 0.0356 \\
  0.0283 & 0.282 & 0.0526 \\
  -0.02 & -0.264 & 0.404 \\
\end{pmatrix}
\begin{pmatrix}
  -0.0144 & 0.0861 & -0.257 \\
  -0.0725 & -0.00926 & -0.102 \\
  0.235 & 0.22 & -0.0195 \\
\end{pmatrix}
\begin{pmatrix}
  -0.258 & -0.144 & -0.298 \\
  -0.0472 & -0.12 & -0.244 \\
  -0.0346 & -0.132 & -0.267 \\
\end{pmatrix}
\]
\[
\begin{pmatrix}
  0.234 & -0.0464 & -0.0475 \\
  -0.233 & -0.0742 & 0.0278 \\
  0.406 & 0.198 & 0.0365 \\
\end{pmatrix}
\begin{pmatrix}
  0.208 & 0.0917 & 0.177 \\
  0.256 & 0.163 & 0.246 \\
  -0.3 & 0.447 & 0.216 \\
\end{pmatrix}
\]
}

Filtro 5
{ \small
\[
\begin{pmatrix}
  0.269 & -0.0167 & 0.119 \\
  0.396 & -0.0312 & -0.147 \\
  0.185 & 0.0514 & -0.0046 \\
\end{pmatrix}
\begin{pmatrix}
  0.162 & 0.185 & 0.526 \\
  0.175 & 0.0848 & 0.25 \\
  0.186 & 0.0925 & 0.00618 \\
\end{pmatrix}
\begin{pmatrix}
  -0.203 & -0.144 & -0.0431 \\
  -0.333 & -0.368 & -0.369 \\
  0.106 & -0.247 & 0.116 \\
\end{pmatrix}
\]
\[
\begin{pmatrix}
  -0.253 & -0.0721 & -0.0373 \\
  -0.00584 & -0.00405 & -0.105 \\
  -0.0279 & 0.195 & 0.101 \\
\end{pmatrix}
\begin{pmatrix}
  0.0943 & 0.1 & -0.203 \\
  0.146 & 0.007 & -0.272 \\
  0.0907 & 0.079 & -0.0317 \\
\end{pmatrix}
\begin{pmatrix}
  0.127 & -0.0332 & -0.045 \\
  0.0486 & -0.0399 & -0.0196 \\
  0.0719 & -0.0504 & -0.186 \\
\end{pmatrix}
\]
\[
\begin{pmatrix}
  -0.137 & -0.216 & -0.176 \\
  -0.198 & -0.257 & -0.48 \\
  0.446 & 0.209 & 0.739 \\
\end{pmatrix}
\begin{pmatrix}
  0.106 & -0.433 & -0.0954 \\
  0.026 & -0.146 & -0.0565 \\
  0.089 & 0.0445 & 0.202 \\
\end{pmatrix}
\]
}

Filtro 6
{ \small
\[
\begin{pmatrix}
  -0.247 & -0.0595 & -0.0535 \\
  0.0435 & -0.172 & -0.0926 \\
  0.208 & 0.075 & -0.0216 \\
\end{pmatrix}
\begin{pmatrix}
  0.0454 & 0.0483 & -0.421 \\
  -0.0344 & -0.00555 & -0.25 \\
  -0.0812 & -0.0348 & 0.108 \\
\end{pmatrix}
\begin{pmatrix}
  -0.0226 & -0.0778 & -0.0793 \\
  -0.275 & -0.106 & 0.0264 \\
  0.222 & 0.118 & 0.168 \\
\end{pmatrix}
\]
\[
\begin{pmatrix}
  -0.344 & 0.083 & -0.46 \\
  -0.0716 & 0.0728 & 0.376 \\
  -0.00233 & -0.0421 & 0.206 \\
\end{pmatrix}
\begin{pmatrix}
  -0.049 & -0.272 & -0.234 \\
  -0.561 & -0.116 & 0.0539 \\
  -0.159 & -0.088 & 0.0225 \\
\end{pmatrix}
\begin{pmatrix}
  0.00748 & 0.104 & 0.129 \\
  0.316 & -0.213 & -0.194 \\
  -0.206 & -0.172 & -0.0363 \\
\end{pmatrix}
\]
\[
\begin{pmatrix}
  -0.362 & -0.188 & -0.109 \\
  -0.164 & -0.14 & 0.248 \\
  0.287 & -0.00438 & -0.0118 \\
\end{pmatrix}
\begin{pmatrix}
  0.0787 & 0.233 & 0.0862 \\
  -0.478 & -0.236 & -0.136 \\
  -0.0943 & -0.296 & 0.0809 \\
\end{pmatrix}
\]
}

\textbf{M7: 3 convoluções $3 \times 3$ aplicadas sobre uma entrada que 
possui 6 canais}

Filtro 1
{ \small
\[
\begin{pmatrix}
  -0.00265 & -0.0219 & -0.0537 \\
  0.0323 & -0.0132 & 0.000866 \\
  -0.00943 & 0.0508 & 0.0912 \\
\end{pmatrix}
\begin{pmatrix}
  -0.0865 & -0.102 & -0.0139 \\
  -0.138 & -0.121 & -0.0187 \\
  -0.333 & 0.14 & 0.207 \\
\end{pmatrix}
\begin{pmatrix}
  0.0701 & 0.119 & 0.0712 \\
  0.1 & 0.114 & 0.0753 \\
  -0.272 & -0.107 & -0.129 \\
\end{pmatrix}
\]
\[
\begin{pmatrix}
  -0.0826 & -0.0981 & -0.0714 \\
  -0.0648 & -0.154 & -0.0278 \\
  -0.00185 & -0.13 & -0.29 \\
\end{pmatrix}
\begin{pmatrix}
  0.0639 & 0.0544 & 0.0949 \\
  0.0222 & -0.0143 & -0.0485 \\
  -0.165 & -0.361 & -0.374 \\
\end{pmatrix}
\begin{pmatrix}
  0.0209 & -0.0422 & 0.0267 \\
  -0.0221 & -0.136 & 0.187 \\
  -0.145 & -0.162 & 0.0329 \\
\end{pmatrix}
\]
}

Filtro 2
{ \small
\[
\begin{pmatrix}
  -0.0417 & 0.0187 & 0.0102 \\
  -0.0852 & -0.0616 & -0.0572 \\
  0.0352 & 0.0446 & -0.0271 \\
\end{pmatrix}
\begin{pmatrix}
  0.127 & 0.0881 & -0.188 \\
  -0.0553 & -0.171 & -0.126 \\
  0.0262 & -0.0522 & -0.0487 \\
\end{pmatrix}
\begin{pmatrix}
  0.115 & 0.113 & -0.0151 \\
  0.00663 & -0.192 & -0.347 \\
  -0.118 & -0.109 & -0.14 \\
\end{pmatrix}
\]
\[
\begin{pmatrix}
  -0.111 & -0.0753 & 0.0229 \\
  -0.0718 & 0.0257 & 0.0466 \\
  -0.0368 & -0.1 & -0.0782 \\
\end{pmatrix}
\begin{pmatrix}
  -0.169 & -0.113 & -0.102 \\
  -0.128 & -0.144 & -0.058 \\
  0.0353 & 0.035 & 0.0523 \\
\end{pmatrix}
\begin{pmatrix}
  -0.0761 & -0.0655 & 0.0406 \\
  -0.208 & -0.0427 & 0.0362 \\
  0.0997 & 0.0213 & 0.166 \\
\end{pmatrix}
\]
}

Filtro 3
{ \small
\[
\begin{pmatrix}
  0.0442 & 0.00305 & 0.0136 \\
  0.199 & 0.195 & 0.227 \\
  0.122 & 0.14 & 0.18 \\
\end{pmatrix}
\begin{pmatrix}
  -0.248 & -0.131 & 0.000851 \\
  -0.121 & 0.104 & -0.168 \\
  0.133 & 0.255 & -0.208 \\
\end{pmatrix}
\begin{pmatrix}
  0.0728 & -0.155 & -0.15 \\
  0.13 & -0.149 & -0.0966 \\
  0.143 & 0.28 & 0.198 \\
\end{pmatrix}
\]
\[
\begin{pmatrix}
  -0.0865 & -0.0534 & 0.0543 \\
  -0.323 & -0.23 & -0.0358 \\
  -0.0141 & -0.00864 & 0.0429 \\
\end{pmatrix}
\begin{pmatrix}
  -0.0336 & 0.0387 & 0.00118 \\
  -0.273 & -0.252 & -0.259 \\
  -0.164 & -0.0251 & -0.288 \\
\end{pmatrix}
\begin{pmatrix}
  -0.0525 & 0.229 & 0.289 \\
  -0.38 & 0.0473 & 0.216 \\
  -0.228 & -0.265 & -0.414 \\
\end{pmatrix}
\]
}

\begin{table}
	\center
	\caption{Tabela com \emph{Bias} aprendidos durante o treinamento}
	\renewcommand{\arraystretch}{1.6}
	\begin{tabular}{c p{8.0cm}}
		\Xhline{6\arrayrulewidth}
		\textbf{Camada} &
			\textbf{Bias} \\
		\Xhline{2\arrayrulewidth}
		M1 & [
			0.09098341  0.06582655 0.07168961  0.04435913  0.88525242
			 0.040886  ] \\
		M2 & [
			0.06287824  0.27774701 0.4061453  -1.30334198 0.06711841
			-0.11221329 0.06533439  1.06334031] \\
		M3 & [
			-0.22721343 -0.17090347 0.11291317 0.15302791 0.08438433
			0.02150897 0.20345436 -0.15001012] \\ 
		M4 & [
			0.71632427 0.06480201 -0.04017521 0.58979249 0.08649875
			0.25791088 -0.08067521 -0.21774861] \\
		M5 & [
			-0.47191876 -0.34982765 0.20474097 0.30674952 -0.0380516
			0.08939933 0.08481776 0.40020576] \\
		M6 & [
			0.57687277 1.21591604 0.21532156 -0.1975698 0.97217804
			1.23713207] \\
		M7 & [
			0.23965664 0.70356482 0.20452037] \\
		
		\Xhline{6\arrayrulewidth}
	\end{tabular}
	\label{tbl:bias}
\end{table}


%\chapter{Titulo do Outro Apêndice}

Isto \'e um exemplo de Apêndice (b).



%Anexos. Use caso necessário. Todos capítulos após o comando \anexo serao tratados como Anexos. 
%\anexo
%\input{src/anexo_a}
%\input{src/anexo_b}

\end{document}
