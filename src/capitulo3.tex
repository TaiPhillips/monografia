%Exemplo de capitulo

\chapter{Trabalhos Relacionados}

\section{Redes neurais convolucionais aplicadas a imagens}

Comparação direta entre diferentes abordagens para reconhecimento e detecção de
imagem podem ser realizadas se um conjunto de dados comum e suficientemente
grande for usado. A competição \emph{ILSVRC, ImageNet Large Scale Visual
Recognition Challenge} é uma competição que ocorre anualmente desde 2010, e é
voltada a este fim. Esta competição permite comparar soluções 
para diferentes probleas, como classificação de imagens e localização de
objetos em um conjunto de
de dados contendo milhões de imagens manualmente rotuladas entre centenas de
categorias. O set de dados está disponível publicamente, e existe um
\emph{workshop} anual relativa a competição do ano. As imagens publicadas são
separados em um set para treinamento, que estão rotulados, e outro set, de
teste, cojutos rótulos não são publicados, e são usados durante a competição
\cite{ILSVRC15}.

O campeão de 2010 foi \cite{lin2010imagenet}, usando extrator de features
HOG e LBP e classificador SVM com 71,8\% de taxa de acerto (\emph{top 5}),
com taxa de classificação de 52,9\%.

O campeão de 2011 na categoria de classificação foi
\cite{perronnin2010large}, empregando vetores
\emph{Fisher} comprimidos e classificador SVM. Na categoria de localização foi
\cite{van2011segmentation}, usando busca selectiva baseada em agrupamento
hierárquico e detecção usando SIFT, RGB-SIFT e SVM como classificador.

Em 2012 houve a primeira vitória de uma submissão baseada em redes neurais
convolucionais. Nas categorias de classificação e localização de imagens o
melhor resultado foi
de \cite{krizhevsky2012imagenet}, usando uma rede neural convolucional contendo
60 milhões de parâmetros e 65.000 neurônios, usando cinco camadas
convolucionais seguidas por camadas \emph{maxpool} e três camadas totalmente
conectadas. Apenas uma terceira categoria, criada naquele ano, denominada
\emph{classificação fina} foi vencida por uma implementação que não envolvia
CNNs.

Em 2013 o vencedor na categoria de detecção usou uma detector de features
customizado baseado em SIFT. Os detalhes não foram divulgados. Na categoria de
classificação o mesmo time que venceu com uma submissão baseada em CNNs, usando
76 milhões de parâmetros. A terceira categoria do ano foi classificação +
localização, vencida por \cite{sermanet2013overfeat} usando CNNs e janelas
deslizantes de escala múltipla.

Em 2014 os organizadores decidiram separar os resultados de acordo com
múltiplos critérios. Um dos resultados vencedores foi
\cite{szegedy2015going}, que é baseada em CNNs, e define uma arquitetura
denominada \emph{inception}, no qual várias técnicas são usadas para reduzir o
número de parâmetros e aumentar o desempenho de classificação, como usar
duas camadas convolucionais $3 \times 3$ ao invéz de uma camada $1 \times 1$. A
rede neural final possui 22 camadas com parâmetros, ou cerca de 100 camadas no
total, usando 1/12 do número de parâmetros usados por
\cite{krizhevsky2012imagenet}.

Em 2015 um dos vencedores foi \cite{he2015deep}, que empregou uma rede neural
com profundidade 152. Segundo o paper, redes neurais com essa profundidade,
quando treinadas da maneira convencional, produzem resultados piores que redes
neurais muito mais razas. Um método foi proposto para fazer o treinamento, que
envolve treinar um subset da rede neural e progressivamente aumentar o tamanho
da rede neural adicionando camadas antes e depois do trecho treinado, sendo que
essas camadas adicionadas estão configuradas para realizarem o equivalente a
uma função identidade, não afetando o resultado do treinamento que já foi
realizado. O método de treinamento foi demonstrado no conjunto de dados
\sigla{CIFAR}{Canadian Institute for Advanced Research} em um modelo
convolucional com 1000 camadas.

\section{Aprendizado de Máquinas para monitoramento de tráfego}

Existem sistemas com variadas capacidades aplicadas a monitoramento e
fiscalização de tráfego. Os sistemas que usam visão computacional
frequentemente aplicam técnicas de aprendizado de máquinas.

Alguns sistemas precisam identificar a placa veicular. Para tal, técnicas de
\sigla{OCR}{\emph{Optical Character Recognition}, reconhecimento ótico de
caracteres} precisam ser empregadas. Muitos sistemas, como
\cite{qadri2009automatic} e \cite{kranthi2011automatic} usam uma sequência que
envolve pré-processamento, localização da placa, segmentação das letras (ou
números) e reconhecimento dos caracteres. Por mais que os primeiros passos não
usem necessariamente aprendizado de máquinas diretamente, elas incluem OCR,
que normalmente normalmente é implementado usando alguma técnica que envolve
treinamento.
No primeiro caso citado o uso é indireto, atravéz de um módulo de software
pré-treinada que o autor usou. No segundo caso o treinamento do sistema de
machine learning foi feito pelo próprio autor.

A solução apresentada em
\cite{kim2000learning} também usa a mesma sequência de
passos para fazer a leitura da placa, porém usa redes neurais para 
fazer a segmentação e SVM para fazer reconhecimento de caracteres. Para a
segmentação o autor usou duas
\sigla{TDNN}{\emph{Time delay neural network}, rede neural de atraso de
tempo}s, que é uma rede neural não convolucional que
fornece para a primeira camada oculta não apenas o valor atual da entrada, mas
também o valor da entrada no processamento em $t-1$, $t-2$, ..., $t-n$. O autor
aplica a rede neural linha por linha nos pixels da rede neural, e outra rede
neural coluna por coluna, processando com um algorítmo escrito manualmente o
resultado das duas redes neurais para obter as regiões envolventes dos
segmentos que contém os dígitos da placa.

Outra área onde estratégias de aprendisado de máquinas pode ser usado é
predição de tráfego. Em \cite{guo2012short} foi proposto um método para fazer
predição em curto prazo para condições normais e anormais de tráfego. A técnica
proposta envolve o uso de \sigla{kNN}{\emph{k-Nearest Neighbour}, k vizinhos
próximos} (\emph{k-Nearest Neighbour}) aplicado aos dados de um
\sigla{SSA}{\emph{Singular Spectrum Analysis}, análise de espéctro singular}
(\emph{Singular Spectrum Analysis}). O SSA é usado para suavisar os dados antes
deles serem fornecidos para o kNN, que faz a precição.

\section{Detecção de placas de veículos}


