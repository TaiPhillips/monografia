%Exemplo de capitulo

\chapter{Trabalhos Relacionados}

Neste capítulo é feita uma apresentação do atual estado da pesquisa na área
deste trabalho. Na sessão \ref{sec:cap3_img} são apresentados trabalhos que
aplicam redes neurais convolucionais em imagens, na \ref{sec:cap3_ml_trafego} 
são trabalhos que aplicam \emph{machine learning} a problemas de tráfego, e na
\ref{sec:cap3_detec_placas} são trabalhos que usam detecção de placas.

\section{Redes neurais convolucionais aplicadas a imagens} \label{sec:cap3_img}

Comparações diretas entre diferentes abordagens para reconhecimento e detecção de
imagem podem ser realizadas se um conjunto de dados comum e suficientemente
grande for usado. A competição \emph{ILSVRC, ImageNet Large Scale Visual
Recognition Challenge} é uma competição que ocorre anualmente desde 2010, e é
voltada a este fim. Esta competição permite comparar soluções 
para diferentes problemas, como classificação de imagens e localização de
objetos em um conjunto de
de dados contendo milhões de imagens manualmente rotuladas entre centenas de
categorias. O conjunto de dados está disponível publicamente, e existe um
\emph{workshop} anual relativo a competição do ano. As imagens publicadas são
separados em um conjunto para treinamento, que está rotulado, e outro
conjunto, de
teste, cujos rótulos não são publicados, e são usados durante a competição
\sigla{ILSVRC}{\emph{ImageNet Large Scale Visual Recognition Challenge},
desafio de reconhecimento de imagens em larga escala ImageNet} \cite{ILSVRC15}.

O campeão de 2010 foi \citeonline{lin2010imagenet}, usando extrator de
\emph{features} HOG e LBP e classificador SVM com 71,8\% de taxa de
acerto (\emph{top 5}), com taxa de classificação de 52,9\%.

O campeão de 2011 na categoria de classificação, o vencedor foi
\citeonline{perronnin2010large}, empregando vetores
\emph{Fisher} comprimidos e classificador SVM. Na categoria de localização foi
\citeonline{van2011segmentation}, usando busca seletiva baseada em agrupamento
hierárquico e detecção usando \sigla{SIFT}{\emph{Scale-invariant feature
transform}, traformada invariante à escala}, RGB-SIFT e SVM como classificador.

Em 2012 houve a primeira vitória de uma submissão baseada em redes neurais
convolucionais. Nas categorias de classificação e localização de imagens o
melhor resultado foi de \citeonline{krizhevsky2012imagenet}, usando uma
rede neural convolucional contendo 60 milhões de parâmetros e 65.000
neurônios, usando cinco camadas
convolucionais seguidas por camadas \emph{maxpool} e três camadas totalmente
conectadas. Apenas uma terceira categoria, criada naquele ano, denominada
\emph{classificação fina} foi vencida por uma implementação que não envolvia
CNNs.

Em 2013 o vencedor na categoria de detecção usou uma detector de
\emph{features}
customizado baseado em SIFT. Os detalhes não foram divulgados. Na categoria de
classificação, o mesmo time venceu com uma submissão baseada em CNNs, usando
76 milhões de parâmetros. A terceira categoria do ano foi classificação +
localização, vencida por \citeonline{sermanet2013overfeat} usando CNNs e
janelas deslizantes de escala múltipla.

Em 2014 os organizadores decidiram separar os resultados de acordo com
múltiplos critérios. Um dos resultados vencedores foi
\citeonline{szegedy2015going}, que é baseada em CNNs, e define uma arquitetura
denominada \emph{inception}, na qual várias técnicas são usadas para reduzir o
número de parâmetros e aumentar o desempenho de classificação, como usar
duas camadas convolucionais $3 \times 3$ ao invés de uma camada $1 \times 1$. A
rede neural final possui 22 camadas com parâmetros, ou cerca de 100 camadas no
total, usando 1/12 do número de parâmetros usados por
\citeonline{krizhevsky2012imagenet}.

Em 2015 um dos vencedores foi \citeonline{he2015deep}, que empregou uma rede
neural com profundidade 152. Segundo o \emph{paper}, redes neurais com essa
profundidade,
quando treinadas da maneira convencional, produzem resultados piores que redes
neurais muito mais rasas. Um método foi proposto para fazer o treinamento, que
envolve treinar uma sessão da rede neural e progressivamente aumentar o tamanho
da rede neural adicionando camadas antes e depois do trecho treinado, sendo que
essas camadas adicionadas estão configuradas para realizarem o equivalente a
uma função identidade, não afetando o resultado do treinamento que já foi
realizado. O método de treinamento foi demonstrado no conjunto de dados
\sigla{CIFAR}{Canadian Institute for Advanced Research} em um modelo
convolucional com 1000 camadas.

\section{Aprendizado de Máquinas para monitoramento de tráfego}
	\label{sec:cap3_ml_trafego}

Existem sistemas com variadas capacidades aplicadas a monitoramento e
fiscalização de tráfego. Os sistemas que usam visão computacional
frequentemente aplicam técnicas de aprendizado de máquinas.

Alguns sistemas precisam identificar a placa veicular. Para tal, técnicas de
\sigla{OCR}{\emph{Optical Character Recognition}, reconhecimento ótico de
caracteres} precisam ser empregadas. Muitos sistemas, como
\citeonline{qadri2009automatic} e \citeonline{kranthi2011automatic} usam uma
sequência que
envolve pré-processamento, localização da placa, segmentação das letras (ou
números) e reconhecimento dos caracteres. Por mais que os primeiros passos não
usem necessariamente aprendizado de máquinas diretamente, elas incluem OCR,
que normalmente é implementado usando alguma técnica que envolve treinamento.
No primeiro caso citado o uso é indireto, através de um módulo de
\emph{software}
pré treinado que o autor usou. No segundo caso o treinamento do sistema de
\emph{machine learning} foi feito pelo próprio autor.

A solução apresentada em
\citeonline{kim2000learning} também usa a mesma sequência de
passos para fazer a leitura da placa, porém usa redes neurais para 
fazer a segmentação e SVM para fazer reconhecimento de caracteres. Para a
segmentação o autor usou duas
\sigla{TDNN}{\emph{Time delay neural network}, rede neural de atraso de
tempo}s, que é uma rede neural não convolucional que
fornece para a primeira camada oculta não apenas o valor atual da entrada, mas
também o valor da entrada no processamento em $t-1$, $t-2$, ..., $t-n$. O autor
aplica a rede neural linha por linha nos \emph{pixels} da rede neural, e outra
rede neural coluna por coluna, processando com um algoritmo escrito manualmente
o resultado das duas redes neurais para obter as regiões envolventes dos
segmentos que contém os dígitos da placa.

Outra área onde estratégias de aprendizado de máquinas pode ser usado é
predição de tráfego. Em \citeonline{guo2012short} foi proposto um método para
fazer
predição em curto prazo para condições normais e anormais de tráfego. A técnica
proposta envolve o uso de \sigla{kNN}{\emph{k-Nearest Neighbours}, k vizinhos
próximos} (\emph{k-Nearest Neighbours}) aplicado aos dados de um
\sigla{SSA}{\emph{Singular Spectrum Analysis}, análise de espectro singular}
(\emph{Singular Spectrum Analysis}). O SSA é usado para suavizar os dados antes
deles serem fornecidos para o kNN, que faz a predição do tráfego.

\section{Detecção de placas de veículos} \label{sec:cap3_detec_placas}

Em \citeonline{luvizonvideo} foi proposto um método para estimar velocidade de
veículos pela detecção e rastreamento de placas veiculares. A etapa de detecção
de placas usa \emph{SnooperText} \cite{minetto.10.icip} configurado
especificamente para esta tarefa. \emph{SnooperText} é um detector de texto
que opera segmentando o texto usando \emph{toggle mapping}, depois filtrando
caracteres usando regras de forma, três descritores e quatro classificadores
SVM, depois agrupados com regras de forma e distância e finalmente
filtrados usando T-HOG e mais um classificador SVM. A detecção de placas ocorre
usando detector de movimentos

\citeonline{anagnostopoulos2008license} faz um \emph{survey} em 2008 de
várias técnicas de reconhecimento de placas de trânsito. Neste artigo
captura em ``tempo real'' é considerado um sistema que processa cada
\emph{frame} em 50ms. Este \emph{survey} encontrou sistemas 
que tipicamente fazem pré processamento, alguns usando imagens binárias,
alguns usando tons de cinza e outros usando cores.

Entre as abordagens que usam imagens binárias encontram-se sistemas baseados em
detecção de bordas que, apesar de muito rápidos, geram
muitos falsos positivos. Um sistema usa \emph{Connected Component Analysis},
que analisa componentes conectados quanto a sua geometria (dimensões e
área) para determinar se cada componente é uma placa de trânsito. O uso
do operador \emph{Snobel} é bastante comum entre os artigos pesquisados para
produzir resultado final dentro do limite de 50ms.

A categoria de abordagens que encontram mais soluções é a que aplica tons de
cinza. Várias abordagens usam
contagem do número de variações abruptas de contraste em um eixo,
tipicamente o horizontal, para localizar placas. Este tipo de algorítimo
pode operar em uma a cada $N$ linhas da imagens, sendo muito econômico em
tempo de CPU, porém é simplista demais para operar em vários cenários.
Processamento estatístico de bordas pode ser usada focando nas letras
para operar bem quando o contorna da placa não é clara. O uso de
\emph{quadtrees} hierárquicos foi proposto, no qual cada quadrante é novamente
dividido se tiver bastante variação de contraste. Segmentos contíguos são
agrupados se o brilho deles for muito diferente ou muito próximo. Cada
segmento pelo seu tamanho e pelo escore dos blocos que o compõe, que
também é baseado em tamanhos. O melhor (ou melhores) \emph{strips} são
selecionados. Este algoritmo é bastante robusto a condições de iluminação e
tem boa taxa de acertos. Uso de janela deslisante, nas quais a média e o
desvio padrão são calculados e usados diretamente contra um limiar
também foram usados com sucesso. \emph{Wavelet transform} foi aplicado para
localizar placas, mas o método se mostrou muito sensível a variações de
distâncias e das características das lentes.

Tentativas de uso das informações de cor utilizadas nos diversos
países foram realizadas, porém não se mostravam estáveis em condições
naturais de iluminação, pelo fato de que a impressão das cores varia de
acordo com a luminosidade. As tentativas envolvem desde classificação
\emph{pixel} a \emph{pixel} das cores até uso de lógica \emph{fuzzy}, redes
neurais com taxas variadas de acerto, de 75\% a 98\%.

O uso de câmeras e iluminação infravermelha foram demonstradas
como sendo capazes de produzir sistemas com taxa de acerto de 99.3\%.
Baixo número de \emph{pixels} foi demonstrado como tendo efeito negativo em
todos os sistemas testados.

Em \citeonline{jain2015automatic} um método de reconhecimento de placas usando
redes neurais é proposto, porém a etapa de localização usa filtro \emph{Sobel}
para detectar bordas e \emph{connected component analysis}.

