%Exemplo de capitulo

\chapter{Conclusões}

Os resultados obtidos mostram que o método proposto é capaz de localizar placas
veiculares com excelentes valores de \emph{precision} e \emph{recall}, enquanto
executa em poucos milissegundos.  
A rede neural reproduz corretamente o valor da função
que está modelando, estimando a posição da placa veicular na sua
entrada, apesar de ter sido treinada com substancial adição de ruído e outras
distorções, demonstrando a validade do método.

O método de particionamento usado para determinar onde a rede neural é
aplicada, não foi suficiente para permitir atingir taxa de \emph{frames}
nominal de nenhum dos vídeos usados no computador testado. O aumento
dessa taxa
pode ser conseguido de várias formas, incluindo o uso de uma GPU mais rápida,
otimização da topologia da rede neural, melhor agendamento de lotes e melhor
distribuição de tarefas para CPU e GPU.

Uma consequência negativa do método de particionamento usado
é a imprecisão na
posição da placa. O algoritmo usado para converter os escores para coordenadas
usou ``máxima local'', e por isso, quando uma placa é localizada, a posição real
dela pode estar $\pm 20 \times \pm 60$ \emph{pixel}, o que pode ser excessivo
para
certas aplicações. No entanto é possível reduzir pela metade com um método mais
sofisticado de conversão dos escores em coordenadas usando os quatro maiores
valores. Se houver \emph{budget} computacional é possível também
aproximar mais as partições, reduzindo arbitrariamente estes valores.

Para futuros trabalhos sugere-se adicionar ao modelo aqui proposto a
capacidade de fazer segmentação e OCR, de forma a produzir um sistema completo
de identificação de placas veiculares. Dois aspectos do método proposto que
podem ser melhorados são a função modelada pela rede neural, que é $C0$ e
poderia ser $C1$ para melhorar a regressão, e o processo de cálculo de
coordenadas das placas a partir dos escores da rede neural que poderia
aumentar a precisão da localização.  Entre os aspectos práticos, e de
implementação, sugere-se continuar o trabalho de otimização da rede
neural para reduzir a quantidade de computações sem gerar muito impacto nos
resultados de classificação.

