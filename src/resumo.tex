%Exemplo de resumo

Esta é uma proposta de método de localização aproximada de placas veiculares
em vídeo desenvolvida usando redes neurais convolucionais aplicadas diretamente
aos valores RGB dos pixels. O treinamento é feito aplicando distorções às
imagens, como
ruído e variação aleatória de brilho, contraste e saturação, para produzir um
modelo robusto, mesmo sem uso de filtros ou normalização. A proposta inclui um
método eficiente de particionar a imagem de forma a reduzir consideravelmente o
número de vezes que a rede neural vai aplicada. Os indicadores \emph{precision}
e \emph{recall} obtidos foram, respectivamente, 98,9\% e 96,7\% para localizar
a placa com erro inferior à $60\text{px} \times 20\text{px}$.
