%Exemplo de resumo

Localização de placas veiculares é um componente importante de sistemas de
controle e fiscalização de tráfego contemporâneos. Alguns sistemas incluem
localização como parte fundamental de um subcomponente. Em alguns casos
o desempenho do sistema de localização pode definir a viabilizada do sistema
como um todo. Neste trabalho, é proposto um método de localização aproximado
de placas veiculares em vídeo desenvolvida usando redes neurais convolucionais
aplicadas diretamente aos valores RGB dos \emph{pixels}. Na implementação aqui
exposta a placa pode ser localizada com uma margem de erro de
$ \pm 20 \times \pm 60$ \emph{pixels}.
O treinamento é feito aplicando distorções às imagens, como
ruído e variação aleatória de brilho, contraste e saturação, para produzir um
modelo robusto, mesmo sem uso de filtros ou normalização. A proposta inclui um
método eficiente de particionar a imagem de forma a reduzir consideravelmente o
número de vezes que a rede neural precisa ser aplicada. Os indicadores
\emph{precision} e \emph{recall} obtidos foram, respectivamente, 98,9\% e
96,7\% enquanto cada quadro de $480 \times 768$ é processado em 67,7 ms em
um notebook Core i5 com GPU GTX 750M.
